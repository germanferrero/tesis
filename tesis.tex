\documentclass[12pt, a4paper, openany, fleqn]{book}
\usepackage{amsmath, amssymb,}
\usepackage[utf8]{inputenc}
\usepackage[spanish]{babel}
\usepackage{relsize}
\usepackage[left=2.00cm, right=2.00cm, top=2.00cm, bottom=2.00cm]{geometry}
\usepackage{xcolor}
\usepackage{dafny}
\usepackage{spverbatim}
\newcommand{\disgrecion}[1]{#1}
\newcommand{\declConst}[2]{\text{const } #1 : #2}
\newcommand{\declVar}[2]{\text{var } #1 : #2}

\renewcommand{\lstlistingname}{Dafny}
\linespread{1.1}
\author{Germán Ferrero}
\title{Construcción formal de programas asistida por Dafny}
\begin{document}
    \chapter{Introducción}
    En el libro Cálculo de Programas (en adelante simplemente ``el libro''), se busca introducir al estudiante en la construcción de programas imperativos a través de un método formal para derivar el programa a partir de su especificación. La misma debe estar definida por una precondición y una postcondición, ambas escritas como fórmulas del cálculo de predicados estudiado en la materia pre correlativa ``Introducción a los Algoritmos''. El método consiste en tender puentes entre los bloques constructivos de un lenguaje imperativo (\textit{if}, \textit{for}, asignación, \textit{skip}, etc) y el cálculo de predicados. De manera tal que uno pueda plantear el esqueleto de un programa de manera creativa, y concluir los detalles razonando - o calculando - sobre el terreno ya conocido del cálculo de predicados.

    En mi opinión el libro tiene algunas características que merecen revisión:
    \begin{itemize}
    \item Utiliza, de manera casi exclusiva, un lenguaje formal que privilegia la especificidad, pero sacrifica legibilidad y facilidad de interpretación.
    \item En ocasiones pasa por alto el origen de las estrategias creativas necesarias para construir programas, introduciendo ``esqueletos'' de programas sin mucho preámbulo, y restringiendo el foco de atención a la derivación de las sentencias que constituyen el resto del programa y sí pueden derivarse mediante el método formal.
    \item Los programas obtenidos y sus pruebas, no están escritos en computadora, y por tanto no pueden ser ejecutados ni verificados de forma automática.
    \end{itemize}

    Nuestra hipótesis, es que resultaría más provechoso para el proceso de aprendizaje, primero, introducir la programación imperativa con ejemplos en un lenguaje no formal. Luego, con el lector y su intuición a bordo, identificar en los ejemplos la utilidad de los conceptos básicos (transformación de estado, precondición, postcondición, invariante, etc) y finalmente proceder a la formalización de los mismos para luego poder probar (formalmente), la corrección de los programas obtenidos y resolver problemas más complejos asistidos por una metodología similar a la del libro pero adaptada al uso de Dafny, un lenguaje de programación con soporte para la declaración de especificaciones, pruebas y un entorno de desarrollo capaz de verificar los programas de manera automática.

    A lo largo de este trabajo desarrollamos esta estrategia alternativa.
    % TODO: Revisar más adelante qué texto resume mejor cada capítulo.%
    [WIP]
    En el capítulo 2 presentamos ejemplos de problemas simples que pueden resolverse con programación imperativa. Apelamos a la intuición y la creatividad para construir los programas e identificamos en ellos los conceptos fundamentales.
    En el capítulo 3 los formalizamos (...)
    En el capítulo 4 realizamos pruebas formales de correctitud utilizando Dafny. (...)[END-WIP]

    \chapter{Introducción a la programación imperativa en Dafny}
    [WIP](...)

    \subsection{Superar un número entero}
    \subsubsection*{La especificación de un programa}

    Queremos obtener un programa que dado un número entero cualquiera nos devuelva otro mayor. Este simple programa, que finalmente resultará en una línea de código similar a esta:
    \begin{verbatim}
        y = x + 1  # Pseudocódigo
    \end{verbatim}
    nos servirá para razonar acerca de qué es un programa imperativo y cómo podemos garantizar su correctitud.

    En primer lugar, necesitamos ``recibir'' o ``leer'' de alguna forma el número en cuestión, para conocer su valor, y luego poder ``devolver'' o ``escribir'' de alguna forma otro valor mayor, una vez que lo calculemos.
    En la programación imperativa el mecanismo que utilizaremos para satisfacer ambas necesidades es el de \textit{transformación de estado}.
    Un \textit{estado} será una colección de nombres de variables y constantes, con sus respectivos valores, a los que el programa tendrá acceso. Cada una de ellas tendrá un tipo asociado (número entero, booleano, cadena de texto, etc). Al \textit{ejecutar} un programa lo que estaremos haciendo es transformar potencialmente ese estado modificando el valor de sus variables.

    En este caso podemos pensar en un programa $S$ que se ejecutará a partir de un estado con una constante $x$ de tipo entero, que aloja el valor de la cantidad a superar y una variable $y$ del mismo tipo donde depositaremos el resultado.

    Ahora bien, ¿Qué podemos asumir de $x$ e $y$ antes de ejecutar el programa? ¿Qué relación esperamos se cumpla entre $x$ e $y$ luego de la ejecución? Respondernos estas preguntas dará resultado a la \textit{especificación} del programa.

    De la descripción informal que enunciamos inicialmente, no se desprende ninguna relación particular entre $x$ e $y$ que deba valer al inicio de nuestro programa (más allá de las propiedades garantizadas por la definición del estado \footnote{Notemos que la definición de estado ya provee garantías intrínsicas: las constantes no modificarán su valor inicial y los valores de las variables y constantes serán del tipo correspondiente.}). Es decir, no hay una precondición en particular.

    Al finalizar el programa, queremos que en el estado el valor de $y$ sea estrictamente mayor a $x$. Ésta será nuestra postcondición.

    En efecto, nuestro programa será especificado por una \textit{precondición} y una \textit{postcondición}. Juntas, funcionan como contrato entre quien desarrolla el programa y quien lo ejecuta. Quien ejecuta el programa debe asegurar la precondición y puede asumir la postcondición al finalizar la ejecución. Quien lo desarolla puede asumir la precondición y debe asegurar la postcondición.

    Nos resultará útil la siguiente notación formal para especificar programas, que utiliza \textit{fórmulas} para escribir la precondición y la postcondición. En este caso:
    \begin{align*}
        & \declConst{x}{\mathbb{Z}}, \declVar{y}{\mathbb{Z}}\\
        & \{P: True\} \\
        & S \\
        & \{Q: y > x\}\\
    \end{align*}

    Primero declaramos las variables y constantes a las que el programa tendrá acceso. Y luego la Terna de Hoare que toma la forma $\{P\}S\{Q\}$, donde $S$ es el programa y $P$ y $Q$ fórmulas que definen la precondición y postcondición respectivamente. La terna se intrepreta así: 
    \begin{center}
        \textit{Cada vez que ejecutemos $S$ a partir de un estado que satisface $P$ llegaremos a un estado que satisface $Q$.}
    \end{center}
    

    En nuestro ejemplo $P$ es $True$, lo que simboliza no tener ninguna precondición en particular, puesto que cualquier estado posible satisface $True$. Y $Q$ es $y > x$.

    Hasta aquí, utilizando los conceptos de estado, fórmulas, y transformación de estado, hemos logrado especificar acabadamente qué debe hacer nuestro programa, aún sin hablar en absoluto de cómo lo hará.

    \subsubsection*{La especificación de un programa en Dafny}

    A lo largo de este trabajo, escribiremos programas formalmente especificados, y verificaremos a través de \textit{pruebas} que sus implementaciones sean correctas, eso es, verificar que efectivamente vale $\{P\}S\{Q\}$.

    Para ayudarnos en esta tarea, utilizaremos Dafny, un lenguaje de programación con soporte para declarar especificaciones, y dotado de un entorno de desarrollo capaz de verificar los programas de manera automática.

    Si bien para programas sencillos, como el de superar un número entero, la correctitud del programa será evidente con solo inspeccionarlo y para otros realizar las pruebas manualmente no será un gran desafío, a medida que construyamos programas más complejos descubriremos la practicidad de poder contar con un verificador automático.

    Escribamos entonces la especificación de nuestro programa en Dafny:

    \dafnyfile{Especificación de "superar un número entero"}{ejemplos/01_superar_especificacion.dfy}

    Hemos utilizado el constructor \inlinedafny{method} de Dafny para especificar nuestro programa. La presencia de $\declConst{x}{\mathbb{Z}}$ en el estado está reflejada por el parámetro \inlinedafny{x:int}, los parámetros de los métodos en Dafny son inmutables. Mientras que $\declVar{y}{\mathbb{Z}}$ está en el estado por la inclusión de \inlinedafny{y:int} en la cláusula \inlinedafny{returns}.
    La cláusula \inlinedafny{ensures y > x} se utiliza para especificar la postcondición.

    \subsubsection*{La asignación}
    Nuestro programa no tiene una implementación aún. La implementación de un programa imperativo estará dada por una sucesión de sentencias de varios tipos (que iremos introduciendo sucesivamente), y transforman el estado de una manera u otra. El primer tipo de sentencia que estudiaremos es la ``asignación''.

    Una asignación, que escribiremos como $x_1,...,x_n := E_1,...,E_n$, donde el término izquierdo es una sucesión de nombres de variables y el derecho un número igual de expresiones, transforma el estado del programa asignando a las variables del lado izquierdo los valores que resultan de evaluar las expresiones correspondientes del lado derecho en el estado actual.

    Dafny utiliza esta misma sintaxis para la asignación. Las variables que aparecen en el lado izquierdo deben haber sido declaradas previamente con una sentencia de declaración de variables, que toma la forma \inlinedafny{var x;}, o estar listadas entre los parámetros o valores de retorno del método.

    Las expresiones del lado derecho serán expresiones de Dafny válidas de cualquier tipo, pero por el momento nos concentraremos en expresiones aritméticas que combinan números y variables con los operadores aritméticos básicos $+,-,*,/,\%$

    Nos resultará natural utilizar la asignación para completar el cuerpo de nuestro programa:
    \dafnyfile{Implementación correcta de "superar un número entero"}{ejemplos/02_superar.dfy}
    Si corremos:
    \begin{verbatim}
        $ dafny verify ejemplos/02_superar.dfy
    \end{verbatim}
    veremos que Dafny automáticamente verifica la correctitud de este programa. Cómo lo hace? Cómo verifica Dafny que a partir de cualquier estado que satisface $True$, luego de ejecutar la asignación \inlinedafny{y := x + 1} se llega a un estado que satisface la postcondición $y > x$?

    Para entender esto, pensemos primero qué garantía puede darnos la ejecución de la asignación acerca del estado resultante. Por su definición, la asignación solo puede garantizarnos que luego de la ejecución se cumple $y = x + 1$, no más. A su vez, esto resulta suficiente para probar, por aritmética, que $y > x$.

    En cambio, si proponemos la implementación:
    \dafnyfile{Implementación incorrecta de "superar un número entero"}{ejemplos/03_superar_incorrecto.dfy}
    la única garantía provista por la asignación, post ejecución, es $y = x * 2$, que no garantiza, para todo estado posible inicial, $y > x$, puesto que para el estado inicial $\sigma:\{y=0,x=0\}$, si bien luego de la ejecución se cumple $y = x * 2 $, no se cumple $y > x$.

    En efecto, si corremos:
    \begin{verbatim}
        $ dafny verify ejemplos/03_superar_incorrecto.dfy
    \end{verbatim}
    Recibiremos como output:
    \begin{spverbatim}
    examples/03_superar_incorrecto.dfy(3,0): Error: a postcondition could not be proved on this return path
    \end{spverbatim}

    \subsubsection*{Precondición más débil y obligaciones de prueba}
    Notemos que la asignación nos exige una precondición mínima, a partir de la cual, podemos asegurar una postcondición $Q$ luego de ejecutar la asignación. En el primer caso era $True$, pues la asignación por si sola aseguraba la postcondición. En el segundo caso $x > 0$ pues de lo contrario no puede asegurarse la postcondición.
    Llamaremos a esta precondición, la \textit{precondición más débil} de la asignación respecto de $Q$ y la denotaremos $wp.(x := E).Q$ (por \textit{weakest precondition} en inglés).

    El primer paso que deber realizarse para intentar la verificación automática es calcular la $wp.(x := E).Q$.
    Esta puede obtenerse mecánicamente mediante la sustitución sintáctica en $Q$ de todas las ocurrencias de $x$ por $E$ (i.e. $Q[x := E]$) gracias al siguiente teorema.

    \textbf{Teorema}: Weakest Precondition de la asignación
    \footnote{Encerraremos una fórmula entre corchetes [] para indicar que es válida para cualquier estado posible}
    \footnote{La prueba puede encontrarse en el Anexo I}
    \begin{align*}
        [ wp.(x:=E).Q \equiv Q[x:=E] ]
    \end{align*}
    Una interpretación útil de este teorema, es que lo único que podemos esperar de la asignación, es que realice la transformación de estado que le corresponde, el resto de la postcondición tiene que poder probarse a partir de la precondición.

    En nuestros ejemplos:
    \begin{align*}
        & wp.(y := x + 1).( y > x )\\
        & \equiv \text{\{ Teorema Weakest Precondition de la asignación \}}\\
        & x + 1 > x\\
        & \equiv \text{\{ aritmética \}}\\
        & True\\
    \end{align*}

    \begin{align*}
        & wp.(y := x * 2).( y > x )\\
        & \equiv \text{\{ Teorema Weakest Precondition de la asignación \}}\\
        & x * 2 > x\\
        & \equiv \text{\{ aritmética \}}\\
        & x > 0\\
    \end{align*}


    El segundo paso es demostrar que la precondición del programa implica la $wp$.
    Siguiendo con los ejemplos, demostrar $True \Rightarrow True$ en un caso y $True \Rightarrow x > 0$ en el otro.
    Llamaremos a estas fórmulas: obligaciones de prueba.

    Sabemos que no existe un método que pueda probar la validez (o refutar) una fórmula dada cualquiera\footnote{En efecto la mayor parte de las teorías son \text{indecidibles}}. Sin embargo existen herramientas, como los SMT Solvers (Satisfiability Modulo Theory Solvers), que pueden intentar o bien probar su validez o bien encontrar un contrajemplo \cite{10.1007/978-3-030-99524-9_23}. Dafny utiliza SMT Solvers para intentar probar las obligaciones de prueba de manera automática. Si lo logra, entonces estamos seguros de la correctitud del programa. Si no lo logra, puede que el programa sea incorrecto (e incluso contemos con un contraejemplo), o bien que Dafny necesite asistencia en la prueba. En tal caso, Dafny nos permite introducir manualmente nuevos lemas y pruebas, que luego el SMT Solver puede aprovechar para verificar nuestro programa.

    \subsection{Concatenación}
    La concatenación es la construcción del lenguaje imperativo que nos permite componer un programa a partir de dos sentencias $S$ y $T$, que se ejecutarán una atrás de la otra.
    Escibimos $S;T$ para denotar el programa que ejecuta primero la sentencia $S$ y luego la sentencia $T$.
    Ejemplo:

    \chapter{El pipeline de verificación de programas Dafny}

    Dafny se apoya en Boogie (un lenguaje intermedio de verificación o IVL, por sus siglas en inglés), para resolver la verificación automática de programas.
    Boogie, como otros IVL, funciona como una capa de abstracción útil en la cual, por arriba, distintos lenguajes con soporte para especificaciones implementan su traducción a Boogie\footnote{
        Boogie fue escrito originalmente como herramienta intermedia del proyecto Spec\#, una extensión de C\# con soporte para especificaciones. Luego tomó vida propia y hoy es utilizado para construir verificadores para otros lenguajes, como el verificador de Dafny para el lenguaje del mismo nombre, o VCC y HAVOC, dos verificadores de C.
    }, y por abajo, Boogie despacha obligaciones de prueba a probadores de teoremas automáticos. En el medio, el IVL, debe encargarse de la generación de esas obligaciones de prueba.

    El pipeline de verificación de programas Dafny se compone entonces, principalmente, por estas tres etapas:
    \begin{itemize}
        \item La traducción del programa Dafny a un programa Boogie.
        \item La generación de obligaciones de prueba a partir del programa Boogie.
        \item La descarga de esas obligaciones de prueba en un SMT Solver, típicamente.
    \end{itemize}

    \section{La traducción de Dafny a Boogie}
    La traducción de Dafny a Boogie no es trivial, la pieza de código que se encarga de ello tiene aproximadamente un millón de líneas escritas en C\#, y codifica cada elemento del lenguaje de Dafny en elementos de Boogie.\footnote{El código fuente para la traducción se encuentra en el \href{https://github.com/dafny-lang/dafny/tree/v4.7.0/Source/DafnyCore/Verifier}{directorio ``Verifier''} del repositorio de Dafny, en donde la clase ``BoogieGenerator'' se construye modularmente a través de múltiples archivos. Dafny además debe encargarse de garantizar trazabilidad entre el código fuente original y los resultados de la verificación obtenidos, a fin de posibilitar un reporte útil de errores.} La expresividad de Boogie posibilita la traducción de lenguajes con múltiples features como Dafny.

    Utilizando la línea de comandos de Dafny podemos obtener la traducción a código Boogie de la siguiente manera
    \begin{verbatim}
        dafny verify my_program.dfy --bprint:my_program.bpl
    \end{verbatim}
    Al hacerlo, nos encontraremos con un programa Boogie mucho más extenso que nuestro programa original en Dafny. Esto es porque la traducción empieza con un preludio\footnote{El código fuente del preludio se encuentra apartado en el \href{https://github.com/dafny-lang/dafny/blob/v4.7.0/Source/DafnyCore/DafnyPrelude.bpl}{directorio ``Prelude''} del repositorio de Dafny}, en el cuál se axiomatizan en Boogie todos los elementos de Dafny, (tipos, expresiones, etc), seguido de la traducción en sí de nuestro programa Dafny.

    \section{La generación de obligaciones de prueba}
    La generación de obligaciones de prueba consiste en una serie de transformaciones del programa Boogie inicial, hacia una versión tal que permita la construcción de una fórmula lógica a partir de las sentencias del programa de forma directa.

    Este pipeline fue descripto por Barnett y Leino\cite{10.1145/1108792.1108813} y se recomienda su lectura para entenderlo en detalle. Aquí haremos un repaso de los conceptos principales que allí se describen.

    \subsection*{ El lenguaje no estructurado de partida}
    El pipeline se basa en programas no estructurados que siguen la siguiente gramática:
    \begin{align*}
        Program \;\;&::=\;\; Block^{+} \\
          Block \;\;&::=\;\; BlockId :\; Stmt;\;\textbf{goto } BlockId^{*} \\
           Stmt \;\;&::=\;\; VarId := Expr\;|\;\textbf{havoc } VarId \\
                &\;\;\;\;\;\;\;\;|\ \textbf{assert } Expr\;|\;\textbf{assume } Expr \\
                &\;\;\;\;\;\;\;\;|\ Stmt ; Stmt \;|\; \textbf{skip} \\
    \end{align*}
    En este pequeño lenguaje un programa está compuesto por uno o más bloques. Al primero de ellos lo denominamos $Start$. Cada bloque está identificado por un $BlockId$, tiene un cuerpo compuesto por una o más sentencias y un conjunto de cero o más bloques \textit{sucesores}. Cuando se ejecuta un bloque se, ejecuta primero su cuerpo, y luego se continúa arbitrariamente con alguno de los bloques sucesores. 

    El operador ``$:=$'' denota la asignación, $\textbf{havoc}$ la asignación de un valor arbitrario a una variable. El operador ``$;$'' se utiliza para componer. Las sentencias $\textbf{assert}$ y $\textbf{assume}$ tienen un rol clave en la generación de obligaciones de prueba y $\textbf{skip}$ es simplemente un atajo para $\textbf{assert True}$.

    La transformación de programas estructurados a esta gramática se logra reemplazando ciclos con $\textbf{goto}$'s y condicionales de la forma:
    \begin{align*}
        \textbf{if}\ (E)\ \{S\}\ \textbf{else}\ \{T\}
    \end{align*}
    con:
    \begin{align*}
        Start:&\;\;\;\textbf{skip};\ \textbf{goto}\ Then,\ Else \\
        Then:&\;\;\;\textbf{assume}\ E;\ S;\ \textbf{goto}\ End \\
        Else:&\;\;\;\textbf{assume}\ \lnot E;\ T;\ \textbf{goto}\ End \\
        End:&\;\;\;...
    \end{align*}
    Las precondiciones pueden codificarse incorporando sentencias $\textbf{assume}$ al principio del bloque $Start$ y las postcondiciones como sentencias $\textbf{assert}$ al final de los bloques sin sucesores.
    Cada programa da lugar a un conjunto de trazas de ejecución posibles empezando desde el bloque $Start$. Un programa es correcto si ninguna de ellas contiene una sentencia $\textbf{assert}$ que evalúe a $False$.

    \subsection*{La eliminación de loops}
    A partir de una versión no estructurada del programa siguiendo esta gramática, los ciclos quedan codificados en la forma:
    \begin{align*}
        LoopHead:\;\;\;&\textbf{assert}\ \text{loop invariant};\\
                       &\textbf{goto}\ Body,\ After \\
        Body:\;\;\;&\textbf{assume}\ \text{loop guard};\\
                   &S;\\
                   &\textbf{goto}\ LoopHead \\
        After:\;\;\;&\textbf{assume}\ \lnot \text{loop guard};\\
                    & ...
    \end{align*}

    A continuación se eliminan los loops, eliminando los ``back edges'' (introducidos por las sentencias de tipo $\textbf{goto}\ LoopHead$). Para esto, primero se mueve la aserción del invariante al bloque predecesor de $LoopHead$, reemplazándolo por sentencias $\textbf{havoc}$ sobre las variables que pueden ser modificadas por el loop, seguido de una sentencias $\textbf{assume}$ para el invariante. De esta forma las trazas que pasen por el bloque $Body$ representan cualquier iteración arbitraria del ciclo y la sentencia $\textbf{goto}\ LoopHead$ puede eliminarse del cuerpo.
    Notar que esta transformación denota una fuerte dependencia en la definición de invariantes para los ciclos que, de no ser provistos por el usuario, se generan automáticamente.

    \subsection*{La pasificación o eliminación de las asignaciones}
    Una vez eliminados los loops, se realiza la ``pasificación'' del programa que consiste en eliminar las asignaciones y en su lugar utilizar sentencias $\textbf{assume}$. Para ello primero se introducen variables auxiliares para cada ``encarnación'' o asignación de las variables del programa, asegurando que en cualquier traza de ejecución posible cada variable sea asignada una única vez, lo que permite a continuación transformar estas asignaciones únicas en sentencias $\textbf{assume}$.

    \subsection*{La construcción de la obligación de prueba}
    Concluídas estas dos transformaciones, se obtiene un programa compuesto por bloques de la forma:
    \begin{verbatim}
        A: S; goto ...
    \end{verbatim} 
    Donde $S$ solo puede ser $\textbf{assume}$, $\textbf{assert}$ o composición de estos dos.\footnote{Las sentencias de tipo $\textbf{havoc }$ pueden eliminarse a los fines generar la obligación de prueba, pues ló único que interesa sobre las variables son sus condicionantes expresados en las sentencias de tipo $\textbf{assume}$ y $\textbf{assert}$. Las sentencias de tipo $\textbf{skip}$, recordemos, son un atajo a $\textbf{assert}\ True$.}
    La Weakest Precondition de cada una de estos tipos de sentencias puede definirse como:
    \begin{align*}
        wp(\textbf{assert}\ P,\;Q)   &= P \land Q \\
        wp(\textbf{assume}\  P,\; Q) &= P \Rightarrow Q \\
        wp(S;T,\;Q)                  &= wp(S,\;wp(T,\;Q)) \\
    \end{align*}
    Para cada bloque A, se define la variable auxiliar $A_{ok}$. Intuitivamente $A_{ok}$ es $true$ si el programa está en un estado a partir del cual todas las ejecuciones que empiecen desde A son correctas. Se postula la ``ecuación de bloque'' para definir $A_{ok}$ formalmente:
    \begin{align*}
        A_{ok} \equiv wp(S, \bigwedge_{B \in Succ(A)} B_{ok})
    \end{align*}
    Donde $Succ(A)$ es el conjunto de bloques sucesores de $A$.

    Por último, siendo $R$ la conjunción de las ecuaciones de esta forma aportadas por cada bloque del programa, la obligación de prueba del programa queda definida como:
    \begin{align}
        R \Rightarrow Start_{ok} \label{eq:VC}
    \end{align}

    Se puede ver, siguiendo los detalles que, si el probador de teoremas resuelve \textit{sat} para la negación de la fórmula \ref{eq:VC}:
    \begin{align*}
        R \land \lnot Start_{ok}
    \end{align*}
    Estamos en situación de que existe un estado a partir del cuál, establecida la semántica del programa ($R$), hay una traza de ejecución incorrecta, es decir que el programa no cumple con su especificación.

    \subsection*{Conclusión}

    La generación de obligaciones de prueba que realiza Boogie, busca crear fórmulas lógicas que sean acotadas en tamaño, y eviten redundancia para facilitar la tarea del probador de teoremas.
  
    Utilizando la línea de comandos de Boogie podemos inspeccionar las distintas versiones del programa obtenidas tras realizar cada una de las etapas del pipeline con la opción \verb|\traceverify|
    \begin{verbatim}
        boogie /traceverify ejemplos/my_program.bpl
    \end{verbatim}

    \section{La delegación de las obligaciones de prueba}
    Las obligaciones de prueba, que resultan del proceso descripto en la sección anterior, deben ser despachadas al probador de teoremas.
    Con la opción \verb|/proverLog| podemos ver el registro de la consulta realizada por Boogie al probador de teoremas, en formato de expresiones SMT-LIB y, como comentarios, las respuestas obtenidas:
    \begin{verbatim}
        boogie /proverLog:my_program.smt my_program.bpl
    \end{verbatim}


    \subsection{El valor absoluto de un número entero}
    [WIP]: (Introduccion del if)

    \subsection{La división de números enteros}
    Queremos obtener un programa que dados dos números $x$ e $y$, compute el cociente $q$ y el resto $r$ de la división entera de $x$ por $y$.
    Recordemos que el cociente y el resto cumplen:
    $$x = y * q + r$$
    Ahora bien, ¿cómo podemos computar $q$ y $r$, para cualquier $x$ e $y$ dados?
    Una alternativa razonable es construir iterativamente tanto $q$ como $r$, partiendo de valores iniciales y acercándolos en cada paso a su valor final.
    Si en todo momento, mantenemos la cantidad de $x$ distribuída entre cierta cantidad de $y$’s (esto es $q$) y el resto que falta para llegar a cubrir $x$ (esto es $r$), entonces podemos imaginar un ciclo en el que empezamos con $q$ igual a 0 y todo $x$ está en $r$.
    Luego en cada paso, y mientras podamos extraer una cantidad $y$ de $r$, la sustraemos de ahí y la pasamos a $q$.
    Es decir, empezamos con:
    \begin{verbatim}
        q:=0;
        r:=x;
    \end{verbatim}
    Y en cada paso mientras que $r \geq y$, ejecutamos
    \begin{verbatim}
        q:=q + 1;
        r:=r - y;
    \end{verbatim}

    Y así sucesivamente hasta que $y$ no quepe en $r$ ( $ y > r $ ). Para entonces $q$ y $r$ serán respectivamente el cociente y el resto de la división de $x$ por $y$.

    Nuestro programa final será:
    \begin{verbatim}
    q:=0;
    r:=x;
    do r >=y
        q:=q+1;
        r:=r-y;
    od
    \end{verbatim}
    Notemos en este programa simple, que:
    \begin{itemize}
        \item En todo momento se cumple:
        $$x = q * y + r$$
        Este es nuestro \textit{invariante}, y proviene de nuestra estrategia inicial de tener distribuído $x$, paso a paso, entre $q$ y $r$.
        \item Que debemos continuar iterando mientras que $r >= y$,
        Esta es nuestra \textit{guarda}.
        \item Que si se cumple el invariante y la guarda deja de cumplirse entonces hemos terminado.
        \item Y que a su vez, estamos seguros de que en algún momento la guarda dejará de cumplirse, por que en cada paso reducimos $r$.
    \end{itemize}

    Este desarrollo alternativo, pone foco en presentar al lector una estrategia de resolución iterativa del problema, acudiendo a la experiencia común entre las personas de ``pasar cosas'' de un lugar a otro, y cómo esa estrategia puede traducirse en un programa concreto.
    La esperanza, es que al finalizar la explicación, el lector esté en mejores condiciones de identificar otros problemas que puedan resolverse programando ciclos, y la importancia que el invariante adquiere como sostén de la estrategia iterativa de resolución.
    \chapter{Dafny}
    Continuará.

    \bibliography{References}
    \bibliographystyle{plain}
\end{document}
