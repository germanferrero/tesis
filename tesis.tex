\documentclass[12pt, a4paper, openany, fleqn]{book}
\usepackage{amsmath, amssymb,}
\usepackage[utf8]{inputenc}
\usepackage[spanish]{babel}
\usepackage{relsize}
\usepackage[left=2.00cm, right=2.00cm, top=2.00cm, bottom=2.00cm]{geometry}
\usepackage{xcolor}
\usepackage{dafny}
\usepackage{spverbatim}
\usepackage{tikz}
\usepackage{pst-poker}
\usepackage{tcolorbox}
\usepackage{titlesec}
\usepackage{titling}
\usepackage{csquotes}
\usepackage[perpage,para,symbol*,bottom,belowfloats]{footmisc}

\newenvironment{abstract}[2]{
  \thispagestyle{empty}  % No page number
  \vspace*{\fill}
  \begin{center}
    {\bfseries \Large Resumen}
    \begin{quotation}
      #1
    \end{quotation}

    \vspace{1in}
    {\bfseries \Large Abstract}
    \begin{quotation}
      #2
    \end{quotation}
  \end{center}
  \vspace*{\fill}
  \newpage
}

\titlespacing*{\chapter}{0pt}{-10pt}{20pt} % Adjust space above the rule
\titleformat{\chapter}[display]
  {\normalfont\huge\bfseries}
  {} % Aquí puedes dejar vacío o poner el número del capítulo si lo quieres
  {0pt}
  {\titlerule\vspace{1ex}\centering}
  [\vspace{1ex}\titlerule]


\tcbuselibrary{breakable}

\newtcolorbox{warningbox}[1]{colback=orange!5!white,colframe=orange!90!black,fonttitle=\bfseries,title=#1}

\newtcolorbox{hintbox}[1]{colback=blue!5!white,colframe=blue!90!black,fonttitle=\bfseries,title=#1}

\newtcolorbox{whitebox}[1][]{
    colback=white,
    colframe=white, % Frame color
    arc=0pt,        % No rounded corners
    leftrule=1mm,   % Thickness of the left rule
    rightrule=0pt,  % No right rule
    toprule=0pt,    % No top rule
    bottomrule=0pt, % No bottom rule
    boxsep=0.1mm,     % Space between text and box
    left=4pt,       % Space between left rule and text
    right=0pt,       % Space between left rule and text
    top=0pt,        % Space between top rule and text
    bottom=0pt,      % Space between bottom rule and text
    beforeafter skip=0pt,
    breakable,
    #1
}

\newtcolorbox{greenbox}[1][]{
    colback=white,
    colframe=green, % Frame color
    arc=0pt,        % No rounded corners
    leftrule=1mm,   % Thickness of the left rule
    rightrule=0pt,  % No right rule
    toprule=0pt,    % No top rule
    bottomrule=0pt, % No bottom rule
    boxsep=0.1mm,     % Space between text and box
    left=4pt,       % Space between left rule and text
    right=0pt,       % Space between left rule and text
    top=0pt,        % Space between top rule and text
    bottom=0pt,      % Space between bottom rule and text
    beforeafter skip=2ex,
    breakable,
    #1
}

\newtcolorbox{redbox}[1][]{
    colback=white,
    colframe=red, % Frame color
    arc=0pt,        % No rounded corners
    leftrule=1mm,   % Thickness of the left rule
    rightrule=0pt,  % No right rule
    toprule=0pt,    % No top rule
    bottomrule=0pt, % No bottom rule
    boxsep=0.1mm,     % Space between text and box
    left=4pt,       % Space between left rule and text
    right=0pt,       % Space between left rule and text
    top=0pt,        % Space between top rule and text
    bottom=0pt,      % Space between bottom rule and text
    beforeafter skip=0pt,
    breakable,
    #1
}

\newif\ifUsePstPoker
\UsePstPokertrue

\newcommand{\disgrecion}[1]{#1}
\newcommand{\declConst}[2]{\text{const } #1 : #2}
\newcommand{\declVar}[2]{\text{var } #1 : #2}
\newcommand{\wip}[1]{\begin{center} [WIP]#1[END WIP] \end{center}}
\renewcommand{\lstlistingname}{Dafny}

\newcounter{example}[chapter]
\renewcommand{\theexample}{\thechapter.\arabic{example}}

% Command to create an example with the desired format
\newcommand{\example}[1]{
  \refstepcounter{example} % Increment the example counter
  \subsubsection*{Ejemplo \theexample: #1}
}

\newcommand{\hoare}[3]{\ensuremath{[#1]\ #2\ [#3]}}
\newcommand{\hoareTheorem}[3]{\ensuremath{[#1]\ #2\ [#3]}}
\newcommand{\hoareTheoremBkp}[3]{\ensuremath{\vdash[#1]\ #2\ [#3]}}
\newcommand{\verticalHoare}[3]{
    \begin{align*}
        &[#1]\\
        &#2\\
        &[#3]
    \end{align*}
}

\newcommand{\inferenceRule}[2]{
    \begin{equation*}
        \frac{#1}{#2}
    \end{equation*}
}

\linespread{1.1}
\author{Germán Ferrero}
\title{Construcción formal de programas asistida por Dafny}
\begin{document}
    \begin{titlepage}
        \centering
        \vspace*{1in}  % Adjust vertical space from top of the page
        
        % Title
        {\Huge \bfseries \thetitle}
        \vspace{0.5in}  % Adjust space between title and next line
        
        % Author
        {\Large \theauthor}
        \vspace{0.25in}
        
        % Director
        {\large Trabajo final de la carrera Licenciatura en Ciencias de la Computación, FaMAF, UNC.}\\
        \vspace{0.25in}
        {\large Dirigido por: Dr. Miguel Pagano} \\
        \vspace{0.25in}

        % Date
        {\large \today} \\
        
        \vfill  % Push the logo to the bottom
        
        % Logo at the bottom
        \includegraphics[width=\textwidth]{assets/logo_famaf_unc.png}  % Adjust the width as necessary
    \end{titlepage}
    \begin{abstract}{
        Analizamos el potencial de Dafny como herramienta de apoyo en la materia Algoritmos y Estructuras de Datos I de la Licenciatura en Ciencias de la Computación. Mediante el uso de esta herramienta, se propone reducir el énfasis en la derivación de programas imperativos desarrollada en lápiz y papel, liberando así recursos cognitivos del estudiantado para enfocarlos en la comprensión de los conceptos fundamentales de verificación formal (pre y poscondición, invariante, función de cota, etc.) y en la construcción, a través del ingenio y la creatividad, de algoritmos a los cuales el estudiantado pueda dar mayor sentido.
    }{
        We analyze the potential of Dafny as a supporting tool in the Algorithms and Data Structures I course of the Bachelor's Degree in Computer Science. By using this tool, we suggest reducing the emphasis on the derivation of imperative programs developed with pen and paper, thereby freeing up students' cognitive resources to focus on understanding the fundamental concepts of formal verification (pre- and postconditions, loop invariants, bound functions, etc.) and on the construction, through inventiveness and creativity, of algorithms that students can better comprehend and relate to.
    }
    \end{abstract}
    \tableofcontents
    \chapter{Introducción}
    El objetivo de esta tesis es analizar el potencial que tiene Dafny como herramienta de apoyo para la materia Algoritmos y Estructuras de Datos I, de nuestra Licenciatura en Cs. de la Computación.

    Esta materia culmina un proceso de formación en métodos formales para la especificación, comprobación y derivación de programas que se inicia con el estudio de lógica proposicional en la materia pre-correlativa Introducción a los Algoritmos y luego se expande hacia la definición de un lenguaje imperativo minimal cuya semántica se define en términos de transformación de predicados.
    En el transcurso, los estudiantes transitan un camino riguroso en el cual aprenden a operar un sistema formal con su alfabeto, expresiones, axiomas y reglas de inferencia, para luego demostrar teoremas dentro del mismo. Este sistema se complejiza paulatinamente con la introducción de predicados y expresiones cuantificadas, hasta el punto en que puede ser utilizado para especificar programas imperativos, en cuanto que la precondición y poscondición de estos programas se expresan como predicados del sistema formal y la semántica de cada una de las sentencias del lenguaje está también definida en base a fórmulas de este sistema.

    Se busca así que los estudiantes conciban un entendimiento profundo de los programas como objetos formales de estudio que deben ser precisamente especificados y cuya implementación puede (o mejor: debe) ser demostrada dentro de un sistema formal para garantizar su corrección.

    Las materias resaltan el valor adicional que se obtiene al especificar y verificar formalmente los programas en comparación con especificarlos vagamente o probar su corrección solo para algunos casos específicos mediante la práctica usual de \textit{testing} de software.

    Consideramos esta enseñanza de la programación a través de métodos formales de altísimo valor, y buscamos con este trabajo potenciarla.

    Durante el cursado de estas materias, y particularmente en Algoritmos y Estructuras de Datos I, la comunidad de estudiantes destina gran parte del tiempo de cursada y práctica fuera del claustro a la formulación de demostraciones dentro del sistema formal aprendido. Un alto porcentaje de los ejercicios de los prácticos de la materia conllevan la demostración de teoremas y es también un condimento central a la hora de las evaluaciones.

    Demostrar teoremas requiere de un entrenamiento meticuloso en la manipulación simbólica, capacidad de abstracción y ejercitación en la búsqueda de patrones para encontrar posibles aplicaciones de los axiomas y reglas de inferencia del sistema formal en cada uno de los pasos de la demostración.
    Esta es una actividad que si bien refuerza nuestro entendimiento de cómo se opera un sistema formal y por qué podemos confiar en él, a menudo satura nuestra capacidad cognitiva, dejando poco espacio para el aprendizaje de otros aspectos, o en otros planos, del tema que estamos estudiando.

    Nos despierta particular interés la puesta en práctica de la manipulación simbólica de la especificación de un programa como metodología para \textit{derivar} su implementación. Mediante esta técnica el programa se construye al mismo tiempo que se construye la demostración de su corrección.
    Esta práctica es tal vez uno de los justificantes del intenso entrenamiento en la operatoria del sistema formal. Justificante en cuanto que se resalta las bondades de la derivación de programas a partir de sus especificación como método para conseguir implementar los programas.

    Nuestra hipótesis de trabajo es que el énfasis actual puesto en la operatoria del sistema formal es tal vez excesivo y contraproducente.

    Demostrar la correción de programas (en el caso de la programación imperativa), implica una comprensión profunda de conceptos fundamentales como tipos de datos, estado y transformación de estado, precondición, poscondición, invariante de un ciclo, función de cota, etc. Incluso también comprender la confianza que depositamos en los axiomas y las reglas de inferencia en cuanto a sus consistencia.
    A la vez, hacerlo con lápiz y papel acarrea un trabajo meticuloso para hilvanar axiomas y reglas de inferencia en la construcción de la demostración, como mencionamos anteriormente.

    Desde nuestra mirada, ambas cosas son importantes, pero lo primero es primordial y lo segundo es complementario. Debemos priorizar el entendimiento profundo de los conceptos fundamentales sin dejar de entender el valor de las pruebas enmarcadas en un sistema formal y la necesidad de construirlas, pero manteniendo el foco en lo primero.

    En la actualidad existen numerosas herramientas de verificación formal de software en las que podemos delegar, en parte, la construcción de pruebas. Estas herramientas nos permiten aprender los conceptos fundamentales, a la vez que resuelven por nosotros el trabajo tedioso, liberando nuestra capacidad cognitiva para dar sentido a los programas que desarrollamos y a las piezas fundamentales que posibilitan su verificación.

    Creemos que en el afán de enseñar las bondades de la derivación formal, propiciamos a menudo el naufragio de muchas y muchos de nosotros que perfectamente hubiéramos podido llegar a construir los mismos programas, mediante otras técnicas, que aprovechen mejor nuestro conocimiento previo, y favorezcan la construcción de sentido sobre los programas que desarrollamos. Técnicas que no necesariamente se basan en la manipulación simbólica, sino en un rango más abierto de posibilidades, como la creatividad y el ingenio en todas sus expresiones, la analogía con otros saberes o la capacidad de resolución de problemas de forma procedural en otros ámbitos de la vida cotidiana puesta al servicio de la escritura de programas. En su tesis doctoral\cite{Losano}, Leticia Losano analiza las dificultades que atraviesan los estudiantes durante el cursado de Introducción a los Algoritmos, en particular en cuanto a la formulación de demostraciones. Su trabajo nos convoca y motiva en la búsqueda de alternativas. Sostenemos además, que estas técnicas alternativas podrían combinarse con la derivación formal en un mismo proceso de desarrollo.

    Finalmente consideramos que, en el primer año de la carrera durante el cual se desarrollan estas materias, delegar parte de la generación de pruebas en las herramientas de verificación de software que tenemos hoy a nuestro alcance puede resultar en que formemos a más estudiantes capaces de obtener software verificado al final del día. Estudiantes que desarrollen un sentido más profundo de los conceptos fundamentales y los pongan en práctica a lo largo del resto de la carrera. E incluso, en un estadio más avanzado de la misma y con otro entendimiento del valor que supone, estén en mejores condiciones para profundizar su conocimiento en sistemas formales.

    \chapter{Fundamentos de programación formal para programar en Dafny}
    \chaptermark{Fundamentos para programar en Dafny}
    En este capítulo presentamos brevemente los conceptos de programación formal que creemos resultan necesarios a la hora de escribir programas imperativos en Dafny.

    \section{Condiciones de contexto para la propuesta}
    La propuesta pedagógica que desarrollamos en este trabajo supone un estudiantado que ya cuente con una primera experiencia en algún lenguaje de programación imperativa durante la cual se hayan familiarizado de manera práctica con los conceptos básicos: tipos de datos, expresiones, precedencia de operadores, árbol de tipado e incluso con la mecánica de los comandos típicos del paradigma: la asignación, la secuencia(;), la alternativa (\textit{if-else}), y el ciclo (\textit{while/for}). Esperaríamos como resultado de esta experiencia que los estudiantes comprendan el control de flujo de un programa imperativo, identifiquen cuándo resulta conveniente utilizar un \textit{if} o un \textit{while}, y hayan escrito baterías de tests no exhaustivas para sus programas.

    A continuación, ubicamos una segunda instancia formativa, de carácter formal, que ofrezca a los estudiantes la posibilidad de profundizar su entendimiento de los programas como objetos formales cuya corrección puede ser demostrada.
    En esta segunda instancia los estudiantes razonan sobre los programas, ya no en un plano mecánico, operativo, sino en el plano semántico, a partir de las definiciones de Hoare.

    En lo que resta de este capítulo repasamos estas definiciones con foco en aquellas que nos provean los elementos necesarios para desarrollar software verificado en Dafny.

    \section{Especificación de un programa}
    La primera pregunta que surge al analizar un programa es tal vez: ¿El programa hace lo que tiene que hacer? Para poder responder esta pregunta primero debemos ser capaces de definir qué debe hacer el programa. Para ello es conveniente contar con un lenguaje suficientemente expresivo y preciso en el cual expresar tal definición. Expresivo en cuanto a que el mismo lenguaje nos permita definir qué hacen distintos tipos de programas, y preciso en cuanto a que la definición no de lugar a ambigüedades.
    A esta tarea la llamamos \textit{especificar} un programa.

    Una forma de especificar qué hace un programa es definiendo condiciones que cumplirán ciertas variables luego de la ejecución del programa. Esta idea da lugar a la siguiente definición de qué es un programa, al menos en cuanto a cómo se relaciona con las variables y constantes a las que tiene acceso.

    En programación imperativa, un programa es un transformador de estados. Un estado es una valuación de las variables y constantes a las que un programa tendrá acceso, y ejecutar el programa, significará una transformación de ese estado inicial, en otro estado final.

    Utilizaremos condiciones sobre las variables y constantes para describir el estado final.
    El lenguaje para expresar estas condiciones será el de la lógica matemática con sus operadores lógicos ($\land$, $\lor$, $\Rightarrow$, $\neg$) y cuantificadores ($\forall$, $\exists$) aplicados a las variables y constantes del programa o a expresiones construidas a partir de ellas. (ej: Para describir el estado final de un programa que implementa la división entera de $X$ por $Y$, diremos que las variables $q$ y $r$ (cociente y resto) satisfacen $X = q * Y + r \land 0 \leqslant r < Y$).

    Notar que si preservamos en constantes (que diferenciamos mediante letras mayúsculas) los valores de entrada de interés, luego en las condiciones sobre el estado final podemos expresar la relación que deben guardar los valores de salida de interés respecto a los de entrada.

    Este mismo lenguaje nos permite especificar condiciones que deben cumplirse en el estado inicial para que se considere válido ejecutar el programa. (ej: Para la división entera tendremos como condición inicial $X \geqslant 0 \land Y > 0$)

    Utilizamos la ``Terna de Hoare'', \hoare{P}{S}{Q} para especificar programas donde $P$ es una precondición, $S$ un programa y $Q$ una poscondición, y la interpretamos así:

    ``Cada vez que se ejecuta $S$ a partir de un estado que satisface $P$, el programa termina en un estado que satisface $Q$''

    \example{Especificación de la división entera}
    \label{esp:division-entera}

    Un programa $S$ que ejecute la división entera de dos números enteros positivos $X$ e $Y$, puede ser especificado con la siguiente terna:
    \verticalHoare{X \geqslant 0 \land Y > 0}{S}{X = q * Y + r \land 0 \leqslant r < Y}
    Donde $X$ e $Y$ serán constantes y $q$ y $r$ variables, todas de tipo entero.
    Notar que la precondición impide la división por $0$ y que $q$ y $r$ tomarán respectivamente los valores del cociente y resto de la división de $X$ por $Y$ luego de ejecutar $S$.

    Escribir la especificación de un programa antes de pasar a implementarlo es (o debiera ser) una práctica ineludible, ya que hacerlo nos reasegura que entendemos qué es lo que debemos implementar.
    Al especificar un programa logramos también separar conceptualmente dos aspectos independientes del mismo: su especificación (qué hace) y su implementación (cómo lo hace).
    Gracias a la precisión del lenguaje utilizado para especificar, quienes quieran saber qué hace el programa solo necesitarán leer y entender su especificación, sin necesidad de revisar su implementación para descubrirlo. Esto es una gran ventaja, ya que la especificación de un programa suele ser más concisa que su implementación.

    Ahora que contamos con un lenguaje preciso para describir qué debe hacer un programa, queda responder cómo hacemos para garantizar que efectivamente lo hace.

    \section{Prueba de corrección}
    Con ``programación formal'' nos referimos a garantizar que un programa efectivamente cumple con su especificación mediante la generación de una demostración de corrección bajo un sistema formal de deducción. Es decir, una secuencia de pasos lógicos que logren deducir la corrección del programa a partir de un conjunto mínimo de axiomas y reglas de inferencia.

    Si logramos generar una demostración de \hoare{P}{S}{Q} diremos que la terna es un teorema y lo denotaremos \hoareTheoremBkp{P}{S}{Q}.

    Al igual que un teorema matemático, por ejemplo $\vdash x = x + y * 0$, donde la igualdad vale para cualquier valor de $x$ e $y$, un teorema de la forma \hoareTheoremBkp{P}{S}{Q} significa que la terna vale para cualquier estado inicial que satisfaga $P$. Es decir, a diferencia del \textit{testing} de programas que chequea algunos casos particulares no exhaustivos, una demostración formal probará la corrección del programa para todos los casos posibles. Lo cual incrementa drásticamente la confianza que podemos depositar en los programas.

    Los primeros esfuerzos por definir un sistema formal de deducción para la programación imperativa se condensaron en lo que conocemos como la lógica de Hoare.
    La misma establece axiomas y reglas de inferencia para un lenguaje simple equipado con los comandos esenciales del paradigma imperativo: la asignación, la secuencia, el \textit{if} y el \textit{while}. Repasaremos la lógica de Hoare en seguida con la ayuda de Dafny.

    \section{Herramientas de verificación automática}
    Hilvanar axiomas y reglas de inferencia para lograr una demostración de un programa, incluso de uno simple, es una tarea tediosa. Y es por eso que desde la presentación de la Lógica de Hoare se sostiene un gran esfuerzo de investigación y desarrollo en técnicas de mecanización y automatización de la generación de demostraciones que dio lugar a las herramientas de verificación automática de software con las que contamos actualmente, como Dafny.

    De ellas debemos esperar que oculten la complejidad de los mecanismos que hilvanan axiomas y reglas de inferencia en milisegundos, y que nos permitan escribir software verificado siempre que entendamos los axiomas y reglas que rigen la automatización que realizan.

    Al esconder dicha complejidad, estas herramientas nos facilitan el acceso a la programación formal permitiéndonos ejercitar la especificación de programas y el uso de un sistema deductivo para demostrar su corrección, enfocando el esfuerzo cognitivo en el plano semántico en vez del mecánico.

    A continuación haremos un repaso de la Lógica de Hoare al tiempo que pondremos a prueba Dafny, en cuanto a su capacidad para aplicar los axiomas y reglas de inferencia del sistema. Este ejercicio, nos proveerá del conocimiento necesario para escribir programas reales en Dafny en el siguiente capítulo.

    \section{Lógica de Hoare en Dafny}
    La lógica de Hoare está compuesta por un axioma, el axioma de la asignación, y un conjunto de reglas de inferencia. Para presentar las reglas de inferencia utilizaremos la siguiente notación:
    \inferenceRule{S_1, ..., S_n}{S}
    Que indica que la \textit{conclusión} $S$ puede deducirse a partir de las \textit{hipótesis} $S_1, ..., S_n$, las cuales serán o bien todos teoremas de la lógica de Hoare (en la forma $[P]S[Q]$) o bien una combinación de teoremas de la lógica de Hoare con teoremas de la lógica matemática.

    \subsection{Axioma de la asignación}
    La asignación es el comando que toma la forma $x := E$ donde $x$ es una variable y $E$ es una expresión, que posiblemente contiene la variable $x$, y utilizamos para asignar a $x$ el valor resultante de evaluar $E$ al momento de la asignación.

    Si denotamos con $P[E/x]$ al predicado que resulta de reemplazar sintácticamente en $P$ todas las ocurrencias libres de $x$ por $E$, el axioma de la asignación es:

    \begin{center}
        \hoare{P[E/x]}{x:=E}{P}
    \end{center}

    El axioma nos dice que un predicado con posibles ocurrencias libres de $x$ siempre valdrá luego de asignar $E$ a $x$ si antes de la asignación valía un predicado similar a $P$ pero en donde esas ocurrencias de $x$ son reemplazadas por $E$.

    Podemos escribir un simple programa en Dafny para probar cómo este resuelve verificaciones que implican conocimiento del axioma de la asignación:

    \example{Axioma de la asignación en Dafny}
    \begin{greenbox}
    \begin{dafny}[gobble=8]
        function E(x: int): int
        predicate P(x: int)

        method axioma_de_la_asignacion()
        {
            var x: int := *;
            assume P(E(x));
            x := E(x);
            assert P(x);
        }
    \end{dafny}
    \end{greenbox}

    En dafny una \textit{función} (\inlinedafny{function}) se utiliza para calcular valores a partir de sus parámetros de entrada, no tiene efectos secundarios y puede tener definiciones recursivas. Un \textit{predicado} (\inlinedafny{predicate}) es equivalente a una función cuyo valor de retorno es un booleano. Un \textit{método} (\inlinedafny{method}) podrá tener parámetros de entrada y de retorno, tener efectos secundarios y utilizar en su cuerpo comandos imperativos.

    En el ejemplo definimos una función $E$ sobre una variable de tipo entero $x$ que devuelve a su vez un valor entero.
    Un predicado $P$ sobre una variable de tipo entero.
    Y un método sin parámetros de entrada ni de retorno en donde ponemos a prueba la habilidad de Dafny para utilizar el axioma de la asignación.
    No es obligatorio definir un cuerpo para $E$ y $P$ lo que nos permite en este ejemplo referirnos a una función arbitraria y a un predicado arbitrario.
    En la línea 6 utilizamos \inlinedafny{x := *} para inicializar $x$ a un entero cualquiera.
    En la línea 7 establecemos la hipótesis de que $P$ vale para $E(x)$ haciendo uso del commando \inlinedafny{assume}.
    En la línea 8 realizamos la asignación.
    Y en la línea 9 le exigimos a Dafny probar que $P$ vale para $x$ en ese punto del programa mediante el commando \inlinedafny{assert}.
    Efectivamente Dafny logra verificarlo (lo cual simbolizamos con la línea verde vertical a la izquierda \footnote{El editor de texto Visual Studio Code junto con la extensión de Dafny utiliza una línea verde como esta para denotar las partes del programa que han logrado ser verificadas, cada vez que guardamos el archivo. Ver Figura\ref{fig:screenshot}}), haciendo uso del axioma de la asignación.

    \begin{figure}[h!]
        \centering
        \includegraphics[width=0.7\textwidth,keepaspectratio]{assets/screenshot_dafny.png}  % Specify your image file here
        \caption{Captura de pantalla del ejemplo ``Axioma de la asignación'' en el editor de texto VSCode}
        \label{fig:screenshot}
    \end{figure}

    \subsection{Regla del fortalecimiento de la precondición}
    Esta regla indica que si tenemos un programa que nos lleva de $P'$ a $Q$ entonces ese mismo programa también nos lleva a $Q$ desde cualquier precondición más fuerte que $P'$.
    \inferenceRule{\hoareTheorem{P'}{S}{Q}\ \ P \rightarrow P'}{\hoareTheorem{P}{S}{Q}}

    \example{Regla del fortalecimiento de la precondición en Dafny}
    \begin{greenbox}
    \begin{dafny}[gobble=8]
        predicate P(x: int)
        predicate P'(x: int)
        predicate Q(x: int)

        method S(x: int)
            requires P'(x)
            ensures Q(x)

        method regla_del_fortalecimiento_de_la_precondicion(x: int)
        {
            assume forall x: int :: P(x) ==> P'(x);
            assume P(x);
            S(x);
            assert Q(x); 
        }
    \end{dafny}
    \end{greenbox}
    Aquí hemos utilizado las sentencias \inlinedafny{requires} y \inlinedafny{ensures} para especificar la precondición y poscondición respectivamente del programa $S$. Durante el proceso de verificación del segundo método, donde se invoca a $S$, Dafny asumirá que $S$ cumple con su especificación, sin inspeccionar su implementación. Esto resulta muy útil en la práctica para partir el programa en programas más pequeños, lograr primero una implementación verificada del método principal, postergando la implementación de las partes.

    Para verificar el \textit{assert} de la línea 15. Dafny tuvo que aplicar la regla del fortalecimiento de la precondición a partir de la hipótesis de la línea 12 y de la especificación de $S$.

    De este ejemplo y del anterior, podemos notar que aunque en la teoría solemos referirnos con un predicado $P$ a una fórmula sobre el conjunto de las variables del programa, en Dafny debemos especificar sobre cuales opera específicamente, lo cual naturalmente facilita el trabajo de la herramienta a la hora de verificar.

    \subsection{Regla del debilitamiento de la poscondición}
    La regla del debilitamiento de la poscondición nos dice que si un programa $S$ nos lleva de $P$ a $Q$, el mismo programa nos lleva de $P$ a cualquier poscondición más débil que $Q$. Podemos ver a Dafny utilizando esta regla con un ejemplo análogo al anterior.
    \inferenceRule{\hoareTheorem{P}{S}{Q'}\  Q' \rightarrow Q}{\hoareTheorem{P}{S}{Q}}

    \example{Regla del debilitamiento de la poscondición en Dafny}
    \begin{greenbox}
    \begin{dafny}[gobble=8]
        predicate P(x: int)
        predicate Q(x: int)
        predicate Q'(x: int)

        method S(x: int)
            requires P(x)
            ensures Q'(x)

        method test(x: int)
        {
            assume forall x: int :: Q'(x) ==> Q(x);
            assume P(x);
            S(x);
            assert Q(x); 
        }
    \end{dafny}
    \end{greenbox}

    \subsection{Regla de la composición}
    La regla de la composición indica que si un programa $S_1$ nos lleva de un predicado $P$ a un predicado $R$, desde el cual otro programa $S_2$ nos lleva hasta un predicado $Q$, entonces componer secuencialmente $S2$ luego de $S1$, nos da un programa que nos lleva del predicado $P$ al predicado $Q$.

    \inferenceRule{\hoareTheorem{P}{S_1}{R}\ \ \hoareTheorem{R}{S_2}{Q}}{\hoareTheorem{P}{S_1;S_2}{Q}}

    \example{Regla de la composición en Dafny}

    El siguiente ejemplo ilustra la capacidad de Dafny para aplicar la regla de la composición.

    \begin{greenbox}
    \begin{dafny}[gobble=8]
        predicate P(x: int)
        predicate Q(x: int)
        predicate R(x: int)

        method S1(x: int)
            requires P(x)
            ensures R(x)
        
        method S2(x: int)
            requires R(x)
            ensures Q(x)

        method regla_de_la_composicion(x: int)
        {
            assume P(x);
            S1(x);
            S2(x);
            assert Q(x);
        }
    \end{dafny}
    \end{greenbox}
    Con frecuencia resultará útil en la práctica insertar un \textit{assert} de algún predicado $R$ en el algún punto intermedio de nuestro programa cuando intuyamos que la verificación debería lograrse aplicando esta regla. Al hacerlo no solo comprobamos que el verificador puede asegurar ese predicado en ese punto, sino que también lo ayudamos a tenerlo en cuenta (ahora como teorema) para probar lo que sigue.

    \subsection{Regla del \textit{if}}
    Esta regla nos dice que si ambas ramificaciones del \textit{if} junto con la información agregada por la correspondiente valuación de la guarda nos llevan de la precondición a la poscondición, entonces el \textit{if}, como constructo que las compone, también lo hace.

    \inferenceRule{\hoareTheorem{P \land B}{S}{Q}\ \ \hoareTheorem{P \land \lnot B}{T}{Q}}{\hoareTheorem{P}{\mathit{if} B\ then\ S\ else\ T}{Q}}

    \example{Regla del \textit{if} en Dafny}

    En este ejemplo Dafny utiliza las especificaciones de los métodos $S$ y $T$ y la regla del \textit{if} para realizar la verificación.

    \begin{greenbox}
    \begin{dafny}[gobble=8]
        predicate P(x: int)
        predicate Q(x: int)
        function B(x: int): bool

        method S(x: int)
            requires P(x) && B(x)
            ensures Q(x)

        method T(x: int)
            requires P(x) && !B(x)
            ensures Q(x)

        method regla_del_if(x: int)
        {
            assume P(x);
            if B(x) {
                S(x);
            } else {
                T(x);
            }
            assert Q(x);
        }
    \end{dafny}
    \end{greenbox}

    En la práctica, para que Dafny pueda utilizar esta regla, debemos asegurarnos de que puede establecer la hipótesis correspondiente a cada rama. Es decir, que si Dafny no está logrando la verificación, debemos inspeccionar cada caso del \textit{if} por separado para ver si la poscondición se cumple al final de su cuerpo.


    \subsection{El \textit{while}, el invariante y la función de cota}
    El $while$ es la sentencia de la programación imperativa que nos permite escribir programas realmente interesantes y creativos, cuya estrategia resolutiva se desenvuelve en uno o más ciclos que computan algún valor deseado. Utilizamos ciclos para encontrar respuestas o construir soluciones (en definitiva, lograr alguna transformación de estado deseada) de manera iterativa y constructiva. En estos programas los ciclos son típicamente la parte más difícil tanto de elaborar como de verificar.

    Para trabajar con ciclos en términos formales debemos abrazar dos conceptos: el invariante del ciclo y su función de cota.
    El invariante es una condición que se cumple a la entrada del ciclo, a la salida del mismo, y durante cada una de las iteraciones. La función de cota es una expresión entera que evalúa a un valor cada vez menor en cada iteración del ciclo y está acotada por abajo asegurando que el ciclo termina eventualmente.

    Si nuestro programa incluye un ciclo que eventualmente termina y realiza algún trabajo interesante, este ciclo tiene un invariante y una función de cota, lo hayamos hecho explícito o no. El invariante del ciclo es su propiedad intrínseca que nos dice que efectivamente estamos construyendo, paso a paso, esa transformación. Y la función de cota es la que nos dice que eventualmente concluiremos el trabajo.

    Imaginemos dos personas que están construyendo una pared cada una y les pedimos que agreguen a su pared una hilada de ladrillos. Para realizar la hilada se necesitan $N$ ladrillos. Una vez que empezaron su tarea, nos detenemos cada tanto a revisar el progreso y por un lado vemos como una de ellas avanza con su columna ladrillo por ladrillo, de izquierda a derecha, mientras que la otra no parece estar siguiendo ningún patrón de avance identificable. Tenemos la certeza de que la primera terminará la hilada por que hay un invariante: a cada momento $t$ la persona ya colocó $t$ ladrillos, y una función de cota: con cada ladrillo colocado, falta uno menos para llegar a $N$.

    Enunciar el invariante y la función de cota incluso antes de implementar el cuerpo del ciclo, nos ayuda a dar sentido a la estrategia resolutiva del mismo, aumenta nuestras posibilidades de implementarlo correctamente y es indispensable para construir una demostración de su corrección.

    Antes de pasar a la regla de inferencia del \textit{while} repasemos el problema de la división entera cuya especificación en términos de pre y pos condición habíamos dado en la sección \ref{esp:division-entera}. Este problema se resuelve con un programa que contiene un ciclo.

    \example{División entera de dos números enteros positivos}

    Queremos obtener un programa que dados dos números enteros positivos $x$ e $y$, compute el cociente $q$ y el resto $r$ de la división entera de $x$ por $y$.
    Recordemos que $q$ y $r$ serán números enteros que cumplen:
    $x = q * y + r \land 0 \leqslant r < y$.
    Ó dicho de otra forma $q$ y $r$ son tales que $x$ se distribuye entre ``$y$ veces $q$'' y un resto $r$ que es menor a $y$.
    Si pensamos el problema como un proceso iterativo en el que debemos repartir la cantidad $x$ entre ``$y$ partes iguales'', podemos decir que al principio no hemos repartido nada aún, y en cada iteración del ciclo repartiremos una unidad a cada una de las partes, mientras alcance para todas. Cuando no alcance para todas, habremos terminado, y tendremos un resto $r$ entre 0 e $y$. En todo momento, lo que aún queda por repartir será el resto $r$, la \textbf{guarda} del ciclo será $r \geqslant y$ (que resulta verdadera si aún alcanza para repartir una unidad a cada una) y lo repartido a cada parte será $q$.

    Antes de entrar al ciclo tendremos $r=x$ y $q=0$ y en todo momento mantendremos como \textbf{invariante} $x = q * y + r \land 0 \leqslant r$, puesto que lo que sacamos de $r$ lo ponemos en $q * y$. Notemos que $r - y$ decrecerá en cada iteración y está acotada por abajo, por lo que nos sirve como función de cota.
    Al final del ciclo valdrá el invariante (por haberlo mantenido a cada paso) y la negación de la guarda (por haber terminado). La conjunción de ambas cosas nos dice que: $x = q * y + r \land 0 \leqslant r < y$. Por tanto $q$ y $r$ serán el cociente y resto de la división entera de $x$ por $y$.

    La implementación en Dafny aún no verificada de nuestro programa luce así:
    \begin{whitebox}[before skip=2ex]
    \begin{dafny}[gobble=8]
        method division_entera(x: int, y: int) returns (q: int, r: int)
            requires x >= 0 && y > 0
    \end{dafny}
    \end{whitebox}
    \begin{redbox}
    \begin{dafny}[gobble=8,firstnumber=3]
            ensures x == q * y + r && 0 <= r < y
        {
    \end{dafny}
    \end{redbox}
    \begin{whitebox}[after skip=2ex]
    \begin{dafny}[gobble=8, firstnumber=5]
            q := 0;
            r := x;
            while r >= y
            {
                q := q + 1;
                r := r - y;
            }
        }
    \end{dafny}
    \end{whitebox}

    \begin{redbox}[after skip=2ex]
        Error: a postcondition could not be proved on this return path.\\
        Could not prove: \inlinedafny{x == q * y + r}
    \end{redbox}

    Para que Dafny pueda verificar esta implementación deberemos \textbf{anotar} el ciclo con un invariante y función de cota apropiados. Ya que, como veremos a continuación, ambos son piezas fundamentales de la regla de inferencia del \textit{while}.


    \subsection{Regla del \textit{while}}

    La regla de inferencia del \textit{while} es la siguiente:

    \inferenceRule{\hoareTheorem{I \land B \land (E = n)}{S}{I \land (E < n)}\ \ \ \ I \land B \rightarrow E \geqslant 0}{\hoareTheorem{I}{while\ B\ do\ S}{I \land \lnot B}}

    La primera hipótesis dice que $S$, el cuerpo del while, debe ser tal que si el invariante $I$ y la guarda $B$ se cumplen antes de ejecutar $S$ luego de ejecutar $S$ el invariante continúa valiendo y además la función de cota $E$ se redujo al menos en una unidad.

    La segunda hipótesis dice que el invariante $I$ en conjunción con la guarda $B$ aseguran que la función de cota es mayor o igual a 0. Estableciendo así una cota inferior para una variable que decrece en cada iteración por la hipótesis anterior.

    Si estas dos hipótesis se cumplen, entonces podemos concluir que estableciendo el invariante en la entrada del ciclo, el mismo nos llevará a un estado en el cual se cumple el invariante y la negación de la guarda.

    Dafny, como lo indica la regla, depende fuertemente de que el invariante del ciclo y la función de cota estén definidos para intentar probar su corrección. Y salvo en casos triviales en los que puede inferirlos por su cuenta\footnote{Para intentar inferir el invariante usará técnicas de aproximación de dominios abstractos\cite{10.1007/11804192_17} que analizan estáticamente el programa para aproximar el rango en el que se mueven sus variables. Para la función de cota, como ejemplos, en caso de que la guarda sea del tipo $E<F$ probará con $F-E$, y en el caso de que sea $E!=F$ probará con $if\ E<F\ then\ F-E\ else\ E-F$.}, será nuestra tarea proveer el invariante del ciclo para que la verificación pueda realizarse.

    Retomando el ejemplo de la división entera, podemos lograr una implementación verificada anotando el ciclo con el invariante $x = q * y + r \land 0 \leqslant r$ (utilizando la anotación \inlinedafny{invariant}), y la función de cota $r$ (con la anotación \inlinedafny{decreases}).

    \begin{greenbox}
    \begin{dafny}[gobble=8]
        method division_entera(x: int, y: int) returns (q: int, r: int)
            requires x > 0 && y > 0
            ensures x == q * y + r && 0 <= r < y
        {
            q := 0;
            r := x;
            while r >= y
                invariant x == q * y + r && 0 <= r
                decreases r - y
            {
                q := q + 1;
                r := r - y;
            }
        }
    \end{dafny}
    \end{greenbox}

    \subsection{Reglas de conjunción y disyunción}
    Dos reglas más completan el sistema deductivo de la Lógica de Hoare para este lenguaje imperativo simple.

    La regla de conjunción de especificaciones:
    \inferenceRule{\hoareTheorem{P}{S}{Q}\ \ \hoareTheorem{P'}{S}{Q'}}{\hoareTheorem{P \land P'}{S}{Q \land Q'}}

    Y la regla de disyunción de especificaciones:
    \inferenceRule{\hoareTheorem{P}{S}{Q}\ \ \hoareTheorem{P'}{S}{Q'}}{\hoareTheorem{P \lor P'}{S}{Q \lor Q'}}

    \subsection{Recapitulación}
    Hemos visto el axioma de la asignación y las reglas de inferencia de la Lógica de Hoare para un lenguaje imperativo simple y pusimos a prueba la capacidad de Dafny para verificar programas que requieren hacer uso de ellas. Al programar en Dafny estaremos constantemente interactuando con el verificador en la tarea conjunta de lograr una implementación verificada de nuestros programas. Dafny intentará hacerlo de manera autónoma, resolviendo el tedio de articular axiomas y reglas en pos de una demostración, pero en ocasiones deberemos ayudarlo, entendiendo qué reglas de inferencia son necesarias a cada paso y verificando que las hipótesis para aplicar esas reglas se satisfacen y el verificador está al tanto de ellas. Además, y siendo esto lo más importante, deberemos anotar cada ciclo con invariantes y funciones de cota.

    \chapter{Metodología para escribir programas en Dafny}
    \chaptermark{Metodología}
    En este capítulo presentamos una metodología para escribir programas en Dafny. La metodología propuesta supone una serie de pasos a seguir que nos resultaron útiles a la hora de ordenar el desarrollo y la interacción con el verificador.
    Introduciremos la metodología a través de un ejemplo concreto: el algoritmo de Euclides para el cálculo del máximo común divisor. Como prólogo de este capítulo, nos gustaría relatar primero la experiencia propia intentando resolver el problema del máximo común divisor antes de ser iluminados con la solución de Euclides.

    \section{¿Qué tan relevante es la derivación de programas? una opinión personal}
    \sectionmark{Derivación de programas: Una opinión personal}
    Queremos escribir un programa, verificado, que compute el máximo común divisor entre dos números enteros positivos $m$ y $n$, con $m \geqslant n$. Formalmente, si utilizamos $x \mid y$ para decir que $x$ divide a $y$, definimos el máximo común divisor $mcd$ como:

    \begin{center}
        \begin{math}
            (mcd \mid m) \land (mcd \mid n) \land (\forall d: (d \mid m) \land (d \mid n) : d \leq mcd)
        \end{math}
    \end{center}

    Nuestra primera exploración en la búsqueda de un algoritmo que compute el máximo común divisor entre $m$ y $n$ fue a partir de la factorización en primos de $m$ y $n$. Dada por:

    \begin{center}
        \begin{math}
            m = \prod_{i=0}^{i=M} p_{i}^{m_i}\ , \ \ \ n = \prod_{i=0}^{i=N} p_{i}^{n_i}
        \end{math}
    \end{center}

    donde $p_0=2$, $p_1= 3$, $p_2= 5$, ... y $p_{M}$, $p_{N}$ son los primos más grandes que aparecen en la factorización de $m$ y $n$ respectivamente; y $m_i$, $n_i$ los exponentes del primo $p_i$ en sus respectivas factorizaciones, (siendo $x_i=0$, si $p_i$ no aparece en la factorización).

    Utilizando estas definiciones el m.c.d. entre ellos será:

    \begin{center}
        \begin{math}
            mcd(m, n) = \prod_{i=0}^{i=min(M, N)} p_{i}^{min(m_i, n_i)}
        \end{math}
    \end{center}

    Proponemos como estrategia de resolución, empezar con la peor solución $mcd' = 1$ e iterar sobre los números primos ($p_0$, $p_1$, ..., $p_{min(M, N)}$) multiplicando $mcd'$ por $p_{i}^{min(m_{i}, n_{i})}$ en cada iteración $i$.

    Si definimos la factorización hasta el $k$-ésimo primo de un número $x$ como:
    \begin{center}
        \begin{math}
            F_k(x) = x * (\prod_{i=k}^{i=X} p_{i}^{x_i})^{-1}
        \end{math}
    \end{center}

    Podemos proponer el siguiente invariante de la estrategia propuesta:
    \begin{align*}
        Inv:&\ \  0 \leqslant i \leqslant min(M, N) \\
            &\land (mcd' \mathrel{|} F_i(m)) \land (mcd' \mathrel{|} F_i(n)) \\
            &\land (\forall d: (d \mathrel{|} F_i(m)) \land (d \mathrel{|} F_i(n)) : d \leq mcd') 
    \end{align*}

    Y probar que la siguiente actualización de $mcd'$ e $i$, mantiene dicho invariante
    \begin{align*}
        & mcd' := mcd' * p_{i}^{min(m_{i}, n_{i})} \\
        & i := i + 1
    \end{align*}

    Para realizar esta actualización necesitaríamos contar con $p_{i}$, $m_{i}$ y $n_{i}$. Para eso podríamos introducir funciones auxiliares que nos provean los primeros $N$ primos y la factorización. Pero nos detenemos aquí en cambio, porque no nos interesa seguir este camino. La resolución propuesta parece correcta, sin embargo su complejidad es mucho mayor a la de la solución propuesta por Euclides hace más de 2300 años.

    Hay alguna técnica de obtención de invariantes que nos hubiera llevado a la misma realización que tuvo Euclides? Puede el método reemplazar a la creatividad?
    Euclides --sospecho-- no llegó a proponer su algoritmo manipulando ecuaciones, sino observando líneas geométricas de longitud discreta. De la misma forma en que Pitágoras llegó a su famoso teorema observando triángulos. De nuevo, sospecho.

    El mismo Dijkstra, en su trabajo \emph{Guarded Commands}\cite[p. 12]{EWD:EWD418} donde presenta su cálculo para la derivación formal de programas reflexiona en las conclusiones (la traducción es mía):

    \blockquote{
        La segunda razón para llevar a cabo estas investigaciones fue mi deseo personal de apreciar mejor qué parte de la actividad de programación puede considerarse una mera rutina formal y qué parte parece requerir ``invención''. Mientras que el diseño de un \textit{if} parece ahora una actividad bastante directa, el diseño de un ciclo requiere lo que yo considero ``la invención'' de una relación invariante y una función variante.(...) Mi presentación de este cálculo no debe ser interpretada como una sugerencia personal de que todos los programas tiene que ser desarrollados de esta manera: solo nos provee una herramienta más.
    }

    En el libro \emph{Cálculo de Programas}, actual bibliografía de referencia de la materia, se reserva un lugar de gran relevancia a la técnica de derivación de programas. Por ejemplo, para el problema del máximo común divisor se plantea derivar el cuerpo del ciclo a partir de las propiedades del $mcd$ en las que se apoya el algoritmo de Euclides.
    Si bien la técnica es ciertamente valiosa, en mi opinión sería más provechoso enfocar la enseñanza en el razonamiento procedural que realizó Euclides y que explica por qué se eligen dichas propiedades del $mcd$ en primer lugar.

    En general creo que podemos quitar relevancia a la derivación de programas como método preferente para construirlos, dar lugar a otros métodos como la creatividad y el ingenio y resaltar en cambio el rol de las técnicas de verificación formal a la hora de \textbf{probar} que los programas son correctos.

    Probar que el algoritmo de Euclides es correcto, es una tarea tan importante como la demostración de cualquier teorema. En ese sentido no hay diferencia entre un programa y una equivalencia matemática. Ambos pueden obtenerse a partir de un proceso creativo, y necesitan ser demostradas, bajo un proceso metódico, cuidadoso y formal.

    Mi siguiente paso fue entonces observar el procedimiento propuesto por Euclides y entender la estrategia resolutiva para luego pasar a construir una implementación verificada en Dafny.

    \section{El algoritmo de Euclides}
    \label{desarrollo_euclides}
    Euclides propuso pensar en $m$ y $n$ como dos varillas conmesurables como se ilustra en la Figura \ref{varillas_euclides}, y buscar la varilla de largo $d$ más larga que logre componer ambas de manera exacta.

    \begin{figure}[h]
        \centering
        \begin{tikzpicture}[yscale=-1]
            % Bar lengths (scaled)
            \def\m{9} % Length of the first bar
            \def\n{4} % Length of the second bar
            \def\scale{1cm} % Scaling factor for bar lengths

            % Draw the bars
            \draw[fill=blue!50] (0, 0) rectangle (\m*\scale, 0.5); % First bar (a)
            \draw[fill=green!50] (\m*\scale, 0) rectangle (\m*\scale+ \n*\scale, 0.5);% Second bar

            \node at (\m*\scale/2, 0.25) {\textbf{m}};
            \node at (\m*\scale + \n*\scale/2, 0.25) {\textbf{n}};

        \end{tikzpicture}
        \caption{Las varillas de Euclides} \label{varillas_euclides}
    \end{figure}

    Que $d$ pueda componer a ambas, es otra forma de decir que $d \mid m$ y $d \mid n$. Que sea la más larga entre las que pueden componerlas quiere decir que $d = mcd(m,n)$.

    Si la más pequeña puede componer a la más grande (i.e. $n \mid m$), entonces ya está: $d=n=mcd(m,n)$, pues cualquier otra varilla que compone a $n$ es necesariamente menor o igual a $n$. Sino, tenemos que $n$ entra alguna cantidad $q$ de veces en $m$ y luego queda un resto $r$ (Figura \ref{varillas_euclides_resto}).

    \begin{figure}[h]
        \centering
        \begin{tikzpicture}[yscale=-1]
            % Bar lengths (scaled)
            \def\m{9} % Length of the first bar
            \def\n{4} % Length of the second bar
            \def\scale{1cm} % Scaling factor for bar lengths

            % Top bars
            \draw[fill=blue!50] (0, 0) rectangle (\m*\scale, 0.5); % First bar (a)
            % Bottom bars
            \draw[fill=green!50] (0, 0.5) rectangle (\n*\scale, 1);
            \draw[fill=green!50] (\n*\scale, 0.5) rectangle (2 * \n*\scale, 1);
            \draw[fill=red!50] (2*\n*\scale, 0.5) rectangle (2 * \n*\scale + \scale, 1);

            \node at (\m*\scale/2, 0.25) {\textbf{m}};
            \node at (\n*\scale/2, 0.75) {\textbf{n}};
            \node at (\n*\scale + \n*\scale/2, 0.75) {\textbf{n}};
            \node at (2 * \n*\scale + \scale/2, 0.75) {\textbf{r}};

        \end{tikzpicture}
        \caption{$n$ no logra componer a $m$, y tenemos un resto $r$} \label{varillas_euclides_resto}
    \end{figure}


    En este caso, la varilla de largo $d$ que estamos buscando debe componer también a $r$. Veamos por qué. Tenemos por un lado que $d$ compone a $n$ y $d$ compone a $m$. Por otro, $m$ está compuesta por una cantidad $q$ de varillas $n$ y un resto $r$.
    Las $q$ varillas de largo $n$ las sabemos compuestas por $d$, por tanto lo que resta ($r$), debe poder componerse con $d$ para que el total $m$ quede compuesto por $d$ a su vez. Si quedara un resto al intentar componer $r$ con $d$, tendríamos en cambio que $d$ no compone $m$ (Figura \ref{varillas_euclides_mcd_del_resto}).

    \begin{figure}[h]
        \centering
        \begin{tikzpicture}[yscale=-1]
            % Bar lengths (scaled)
            \def\m{9} % Length of the first bar
            \def\n{4} % Length of the second bar
            \def\bd{0.8} % Length of falsy mcd bar
            \def\scale{1cm} % Scaling factor for bar lengths

            % Top bars
            \draw[fill=blue!50] (0, 0) rectangle (\m*\scale, 0.5); % First bar (a)
            % Bottom bars
            \draw[fill=green!50] (0, 0.5) rectangle (\n*\scale, 1);
            \draw[fill=green!50] (\n*\scale, 0.5) rectangle (2 * \n*\scale, 1);
            \draw[fill=red!50] (2*\n*\scale, 0.5) rectangle (2 * \n*\scale + \scale, 1);
            \draw[fill=orange!50] (0, 1) rectangle (\bd, 1.5);
            \draw[fill=orange!50] (\bd, 1) rectangle (2*\bd, 1.5);
            \draw[fill=orange!50] (\bd * 2, 1) rectangle (3*\bd, 1.5);
            \draw[fill=orange!50] (\bd * 3, 1) rectangle (4*\bd, 1.5);
            \draw[fill=orange!50] (\bd * 4, 1) rectangle (5*\bd, 1.5);
            \draw[fill=orange!50] (\bd * 5, 1) rectangle (6*\bd, 1.5);
            \draw[fill=orange!50] (\bd * 6, 1) rectangle (7*\bd, 1.5);
            \draw[fill=orange!50] (\bd * 7, 1) rectangle (8*\bd, 1.5);
            \draw[fill=orange!50] (\bd * 8, 1) rectangle (9*\bd, 1.5);
            \draw[fill=orange!50] (\bd * 9, 1) rectangle (10*\bd, 1.5);
            \draw[fill=orange!50] (\bd * 10, 1) rectangle (11*\bd, 1.5);
            \draw[fill=red!75] (\bd * 11, 1) rectangle (\m*\scale, 1.5);

            \node at (\m*\scale/2, 0.25) {\textbf{m}};
            \node at (\n*\scale/2, 0.75) {\textbf{n}};
            \node at (\n*\scale + \n*\scale/2, 0.75) {\textbf{n}};
            \node at (2 * \n*\scale + \scale/2, 0.75) {\textbf{r}};
            \node at (\bd/2, 1.25) {\textbf{d}};
            \node at (\bd * 1 + \bd/2, 1.25) {\textbf{d}};
            \node at (\bd * 2 + \bd/2, 1.25) {\textbf{d}};
            \node at (\bd * 3 + \bd/2, 1.25) {\textbf{d}};
            \node at (\bd * 4 + \bd/2, 1.25) {\textbf{d}};
            \node at (\bd * 5 + \bd/2, 1.25) {\textbf{d}};
            \node at (\bd * 6 + \bd/2, 1.25) {\textbf{d}};
            \node at (\bd * 7 + \bd/2, 1.25) {\textbf{d}};
            \node at (\bd * 8 + \bd/2, 1.25) {\textbf{d}};
            \node at (\bd * 9 + \bd/2, 1.25) {\textbf{d}};
            \node at (\bd * 10 + \bd/2, 1.25) {\textbf{d}};

        \end{tikzpicture}
        \caption{Si $d$ pudiera componer a $n$ pero no a $r$, tampoco compondría a $m$}
        \label{varillas_euclides_mcd_del_resto}
    \end{figure}

    Sea $r = m \% n$ (el módulo), tenemos entonces dos casos: a) $r = 0 \land d = n$ ó b) $r > 0 \land d \mid r $. En el caso b), como $d \mid n$ y $d \mid r$, tenemos que $d \leq mcd(n, r)$. Veamos además que $d = mcd(n,r)$ probando que $d \geqslant mcd(n,r)$. Como $m = q * n + r$, tenemos que $mcd(n, r) | m$ y por tanto $d \geqslant mcd(n,r)$. Por tanto lo tanto $d = mcd(n,r)$.

    Es decir que los dos casos son: a) $r = 0 \land d = n$ ó b) $r > 0 \land d = mcd(n, r) $

    En el caso a) no queda nada por hacer, en el caso b) podemos repetir el razonamiento reemplazando $m$ por $n$ y $n$ por $r$. Y como en cada iteración el resto de la división de $m$ y $n$ siempre será menor a $n$, con el reemplazo propuesto el resto siempre será menor al de la iteración anterior. A su vez, por ser resto de una división, está acotado por abajo en $0$. Por tanto sabemos que en algún momento terminaremos.

    \section{La metodología}
    Implementar el algoritmo de máximo común divisor de Euclides en Dafny de manera tal que el verificador logre verificar nuestra implementación despierta nuevas preguntas y potenciales respuestas sobre \textbf{qué} conocimiento posee el verificador de Dafny y \textbf{cómo o cuándo} puede aplicarlo.

    ¿Dafny sabe acaso que $mcd(m, n) = mcd(n, m\%n)$? Se lo podemos decir? Si se lo decimos, puede aplicarlo en la verificación?
    Las respuestas rápidas son: ``no lo sabe'', ``se lo podemos decir'', y ``hay que ayudarlo a ubicar el nuevo saber en el formato y lugar correcto''.

    Nuestro camino hacia lograr la implementación verificada no fue lineal, sino que tuvo una etapa prematura de probar distintas estrategias ``contra la caja negra''. Sin embargo, el desarrollo puede realizarse de manera metodológica y ordenada, siguiendo estos pasos:
    \begin{enumerate}
        \item Especificar el programa.
        \item Testear la especificación.
        \item Definir el invariante, función de cota y guarda del ciclo.
        \item Declarar las variables necesarias en el método principal.
        \item Especificar inicialización y cuerpo del ciclo.
        \item Implementar la inicialización y el cuerpo del ciclo.
        \item Fortalecer el invariante, en caso de ser necesario.
    \end{enumerate}

    \subsubsection{Especificación del programa}
    Definimos para ello el predicado \textit{es el mcd} que será verdadero si y solo si $d$ es el máximo común divisor de $m$ y $n$ y el método principal \textit{máximo común divisor} con su precondición sobre $m$ y $n$ y su poscondición:

    \begin{greenbox}
    \begin{dafny}[gobble=8]
        predicate es_el_mcd(d: int, m: int, n: int)
        {
            0 < d <= n &&
            m % d == 0 &&
            n % d == 0 &&
            forall d' :: 
                (0 < d' <= n && m % d' == 0 && n % d' == 0)
                    ==> d' <= d
        }

        method maximo_comun_divisor(m: int, n: int) returns (mcd: int)
            requires 0 < n <= m
            ensures es_el_mcd(mcd, m, n)
    \end{dafny}
    \end{greenbox}

    Tener definido el predicado \textit{es el mcd} nos será útil a continuación para \textit{testear} la especificación y definir el invariante.

    \subsubsection{Testear la especificación}
    Al escribir software verificado, restringimos el espacio para el error humano de la implementación a la definición de la especificación.
    Si no detectamos los errores de especificación antes de pasar a la etapa de implementación, es posible que malinterpretemos los errores de verificación como deficiencias en la implementación o en la capacidad del verificador para realizar la prueba de corrección, cuando en realidad el problema está siendo arrastrado desde la especificación misma.
    Afortunadamente en Dafny podemos \textit{testear} la especificación.
    El verificador es capaz de validar el predicado \textit{es el mcd} para algunos casos concretos con números pequeños que nos ayudan a ganar seguridad sobre la corrección de la especificación.\footnote{Se recomienda al lector modificar el predicado o los casos de tests para forzar un error de verificación. En particular ver qué sucede si equivocamos $d' < d$ en vez de $d' <= d$ en el predicado}.

    \begin{greenbox}
    \begin{dafny}[gobble=8]
        method test_es_el_mcd(){
            assert es_el_mcd(1, 8, 3);
            assert !es_el_mcd(2, 8, 3);
            assert es_el_mcd(5, 2345, 5000);
            assert !es_el_mcd(4, 2345, 5000);
            assert es_el_mcd(1, 49163, 9113);
        }
    \end{dafny}
    \end{greenbox}

    \subsubsection{Definir el invariante, función de cota y guarda del ciclo}

    Recordemos que la estrategia algorítmica era, sabiendo que $d$ es tal que $d = mcd(m, n)$, ir seleccionando en cada iteración dos varillas una más larga: $m'$ y otra más corta: $n'$, manteniendo dos afirmaciones:
    \begin{itemize}
        \item $d = mcd(m', n')$ y
        \item sí $n'$ compone a $m'$, entonces $d = n'$.
    \end{itemize}

    Nos será útil definir el invariante como un predicado:

    \begin{greenbox}
    \begin{dafny}[gobble=8]
        predicate invariante(d: int, m': int, n': int) {
            es_el_mcd(d, m', n')
            && (m' % n' == 0 ==> d == n')
        }
    \end{dafny}
    \end{greenbox}

    En la estrategia propuesta, seleccionábamos $m'$ y $n'$ de forma tal que $m'\%n'$ es menor en cada iteración y al llegar a $m'\%n' = 0$ terminábamos. Proponemos entonces como guarda del ciclo $m'\%n'>0$, y como función de cota $m'\%n'$.

    \subsubsection{Declarar las variables necesarias en el método principal}
    Hay tres variables de tipo $int$ implicadas en el invariante: $d$, $m'$, $n'$.
    Las últimas dos: $m'$, $n'$ tomarán valores concretos en la inicialización y serán actualizadas en cada iteración, mientras que para $d$ no conocemos su valor hasta el final. Lo único que sabemos de $d$ es que al iniciar el programa refiere al $mcd$ de $m$ y $n$ (los parámetros de entrada), y luego en cada iteración equivale también al $mcd$ de las varillas que estamos analizando. Esto convierte a $d$ en una variable especial, que es necesaria para la verificación pero no se computará en tiempo de ejecución del programa. En Dafny estas variables se llaman \textit{ghost variables} y se definen así:

    \begin{whitebox}[beforeafter skip=2ex]
    \begin{dafny}[gobble=8]
        ghost var d: int :| p(d);
    \end{dafny}
    \end{whitebox}

    \noindent que leemos como ``Sea d tal que el predicado p vale para d''. Utilizaremos una ghost variable $d$ para referirnos al $mcd$ de $m$ y $n$.

    \begin{whitebox}[beforeafter skip=2ex]
    \begin{dafny}[gobble=8]
        ghost var d: int :| mcd(d, m, n);
    \end{dafny}
    \end{whitebox}

    A continuación escribimos la declaración de estas variables al inicio del método.

    \begin{whitebox}[before skip=2ex]
    \begin{dafny}[gobble=8]
        method maximo_comun_divisor(m: int, n: int) returns (mcd: int)
            requires 0 < n <= m
            // ensures es_el_mcd(mcd, m, n)
        {
    \end{dafny}
    \end{whitebox}
    \begin{redbox}
    \begin{dafny}[gobble=8,firstnumber=5]
            ghost var d: int :| es_el_mcd(d, m, n);
    \end{dafny}
    \end{redbox}
    \begin{whitebox}[after skip=2ex]
    \begin{dafny}[gobble=8,firstnumber=6]
            var m': int, n': int;
        }
    \end{dafny}
    \end{whitebox}


    \begin{redbox}[after skip=2ex]
        Error: cannot establish the existence of LHS values that satisfy the such-that predicate
    \end{redbox}


    Al hacerlo notamos que Dafny devuelve un error para la línea 5 
    \footnote{Aquí comentamos la poscondición a propósito. Ver \href{https://github.com/dafny-lang/dafny/issues/6122}{Issue \#6122 en Dafny}.}.
    Nosotros sabemos que para dos números enteros positivos existe un máximo común divisor pero Dafny no lo sabe. Podemos decírselo definiendo un axioma que establezca que el máximo común divisor entre $m$ y $n$ existe e invocándolo justo antes de la definición de $d$
    \footnote{Un axioma en Dafny es un lema que no conlleva una prueba. Si quitamos :axiom, Dafny nos pedirá que escribamos la prueba en el cuerpo del lema.}.

    \begin{greenbox}
    \begin{dafny}[gobble=8]
        lemma {:axiom} existe_un_mcd(m:int, n:int)
            ensures exists d: int :: es_el_mcd(d, m, n)

        method maximo_comun_divisor(m: int, n: int) returns (mcd: int)
            requires 0 < n <= m
            // ensures es_el_mcd(mcd, m, n)
        {
            assert exists d: int :: es_el_mcd(d, m, n) by {
                existe_un_mcd(m, n);
            }
            ghost var d: int :| es_el_mcd(d, m, n);
            var m': int, n':int;
        }
    \end{dafny}
    \end{greenbox}

    Con la introducción del axioma, el error de verificación desaparece. La construcción \textit{assert ... by} indica que utilizamos ese lema para concluir la condición del assert.

    \subsubsection{Especificar inicialización y cuerpo del ciclo.}

    Cuando invocamos un método auxiliar desde un método principal, para la verificación de este último Dafny asume que el primero cumple con su especificación y se limita a verificar el método principal bajo esa asunción. Haciendo uso de esta decisión de diseño podemos definir métodos con su especificación para la inicialización y cuerpo del ciclo, y utilizarlos en el método principal antes de pasar a implementarlos.

    En la inicialización, requeriremos como precondición la misma precondición sobre los parámetros de entrada que tiene nuestro método principal, sumado a la condición que impusimos sobre la variable ghost $d$, luego pedimos que la inicialización asegure el invariante de modo que se satisfaga al ingresar al ciclo.

    \begin{greenbox}
    \begin{dafny}[gobble=8]
        method inicializacion(ghost d: int, m: int, n: int)
            returns (m': int, n': int)
            requires 0 < n <= m
            requires es_el_mcd(d, m, n)
            ensures invariante(d, m', n')
    \end{dafny}
    \end{greenbox}

    En la especificación del cuerpo pedimos que la guarda y el invariante se cumplan antes de la ejecución y aseguramos que luego de ejecutar el cuerpo el invariante se sigue cumpliendo y que la función de cota se redujo.


    \begin{greenbox}
    \begin{dafny}[gobble=8]
        method cuerpo(ghost d: int, m: int, n: int)
            returns (m': int, n': int)
            requires invariante(d, m, n)
            ensures invariante(d, m', n')
            ensures m' % n' < m % n
    \end{dafny}
    \end{greenbox}

    Ahora, como decíamos, podemos invocar estos métodos en el método principal e intentar obtener una implementación verificada del mismo. Si lo logramos, habremos validado la elección del invariante, la función de cota y la guarda incluso antes de implementar el cuerpo del ciclo y su inicialización.


    \begin{greenbox}
    \begin{dafny}[gobble=8]
        method maximo_comun_divisor(m: int, n: int) returns (mcd: int)
            requires 0 < n <= m
            ensures es_el_mcd(mcd, m, n)
        {
            assert exists d: int :: es_el_mcd(d, m, n) by {
                existe_un_mcd(m, n);
            }
            ghost var d: int :| es_el_mcd(d, m, n);
            var m': int, n':int , r: int;
            m', n' := inicializacion(d, m, n);
            while (m' % n' > 0)
                decreases m' % n'
                invariant invariante(d, m', n')
            {
                m', n' := cuerpo(d, m', n');
            }
            return n';
        }
    \end{dafny}
    \end{greenbox}


    \begin{warningbox}{Introducción de un imposible}
    Cuando Dafny asume que los métodos auxiliares cumplen con la especificación, lo hace incluso si la especificación es inconsistente, como se ilustra en el siguiente ejemplo:

    \begin{greenbox}
    \begin{dafny}[gobble=8]
        method auxiliar (y: int) returns (y': int)
            ensures y' >= 0
            ensures y' < 0

        method principal() {
            var y: int := 1;
            y := auxiliar(y);
            assert false;
        }
    \end{dafny}
    \end{greenbox}

    A partir de un imposible puede probarse cualquier cosa, incluso \textit{false}, como sucede aquí. Cuando tengamos dudas sobre las condiciones que Dafny está logrando verificar, podemos asegurarnos de que no hay inconsistencia introduciendo \inlinedafny{assert false} y chequeando que falla la verificación.
\end{warningbox}

\subsubsection{Implementar la inicialización y el cuerpo del ciclo.}

    Con la siguiente implementación del método de inicialización, la verificación del mismo sale de forma directa, como es de esperar a partir del axioma de la asignación.

    \begin{greenbox}
    \begin{dafny}[gobble=8]
        method inicializacion(ghost d: int, m: int, n: int)
            returns (m': int, n': int)
            requires 0 < n <= m
            requires es_el_mcd(d, m, n)
            ensures invariante(d, m', n')
        {
            m' := m;
            n' := n;
        }
    \end{dafny}
    \end{greenbox}

    Si postulamos para el cuerpo del ciclo la asignación a $m'$ y $n'$ que obtuvimos en el desarrollo de la sección \ref{desarrollo_euclides} veremos que el verificador sobrepasa el límite de tiempo permitido antes de lograr verificar la implementación.

    \begin{whitebox}[before skip=2ex]
    \begin{dafny}[gobble=8]
        method cuerpo(ghost d: int, m: int, n: int)
            returns (m': int, n': int)
            requires invariante(d, m, n)
    \end{dafny}
    \end{whitebox}
    \begin{redbox}
    \begin{dafny}[gobble=8,firstnumber=4]
            ensures invariante(d, m', n')
            ensures m' % n' < m % n
    \end{dafny}
    \end{redbox}
    \begin{whitebox}
    \begin{dafny}[gobble=8,firstnumber=6]
        {
    \end{dafny}
    \end{whitebox}

    \begin{redbox}
    \begin{dafny}[gobble=8,firstnumber=7]
            var r := m % n;
    \end{dafny}
    \end{redbox}

    \begin{whitebox}
    \begin{dafny}[gobble=8,firstnumber=8]
            m' := n;
            n' := r;
    \end{dafny}
    \end{whitebox}
    \begin{redbox}[after skip=2ex]
    \begin{dafny}[gobble=8,firstnumber=11]
        }
    \end{dafny}
    \end{redbox}

    \begin{redbox}[after skip=2ex]
        Verification of 'cuerpo' timed out after 20 seconds. (the limit can be increased using --verification-time-limit)
    \end{redbox}


    Sucede que Dafny no conoce la propiedad $mcd(m, n) = mcd(n, m\%n)$ que es, junto al teorema de la asignación, la clave para poder probar el método. Si introducimos un axioma que establezca esa propiedad Dafny logra utilizarlo efectivamente para verificar la implementación del cuerpo.

    \begin{greenbox}
    \begin{dafny}[gobble=8]
        lemma {:axiom} mcd_del_modulo(d:int, m:int, n:int)
            requires es_el_mcd(d, m, n)
            requires n > 0
            ensures es_el_mcd(d, n, m % n)
        
        method cuerpo(ghost d: int, m: int, n: int)
            returns (m': int, n': int)
            requires invariante(d, m, n)
            ensures invariante(d, m', n')
            ensures m' % n' < m % n
        {
            assert es_el_mcd(d, n, m % n) by {
                mcd_del_modulo(d, m, n);
            }
            var r := m % n;
            m' := n;
            n' := r;
        }
    \end{dafny}
    \end{greenbox}

    La versión final verificada del algoritmo de Euclides para el máximo común divisor, con todos los componentes reunidos, será \footnote{
    Una implementación de este algoritmo utilizando las propiedades de $mcd(m, n) = mcd(m, m - n)$ si $m \ge n$ y la de simetría, puede encontrarse en la serie Dafny Power User de Rustan Leino: \href{https://leino.science/papers/krml279.html}{Case study of definitions, proofs, algorithm correctness: GCD
    }. Leino define la función GCD como el máximo elemento en la intersección de los conjuntos de divisores de $m$ y $n$ y luego utiliza esta función en la especificación del método. Incluye también las pruebas de los lemas necesarios. Se recomienda su lectura como complemento de esta sección.}:

    \begin{greenbox}
        \dafnyfile{Versión final del máximo común divisor en Dafny}{ejemplos/mcd/mcd.dfy}
    \end{greenbox}

    \subsubsection{Fortalecer el invariante}
    Como veremos en próximos ejemplos, hay ocasiones en que el invariante elegido resulta suficiente para establecer la poscondición desde la perspectiva del método principal, pero para la implementación del cuerpo del ciclo requerimos de variables auxiliares con sus propios invariantes que posibilitan establecer el invariante original.
    En estos casos decimos que fortalecemos el invariante y en términos prácticos nos llevará a repetir los pasos 3 a 6, agregando las nuevas variables auxiliares y modificando el invariante para que las contemple.

    \subsubsection{Nota sobre los lemas}
    Nuestro primer intento para los lemas \textit{existe\_el\_mcd} y \textit{mcd\_del\_modulo} fue declararlos de la siguiente manera:

    \begin{greenbox}
    \begin{dafny}[gobble=8]
        lemma {:axiom} existe_el_mcd()
            ensures forall m, n :: 0 < n <= m ==>
                exists d: int :: es_el_mcd(d, m, n)

        lemma {:axiom} mcd_del_modulo()
            ensures forall d, m, n : int :: 0 < d <= n <= m ==>
                (es_el_mcd(d, m, n) <==> es_el_mcd(d, m, m % n))
    \end{dafny}
    \end{greenbox}

    E invocarlos al inicio del método principal. Pero Dafny no logra verificar el programa de esta manera. A esto nos referíamos con la pregunta de \textbf{cómo o cuándo} puede aplicar Dafny los lemas y axiomas que le proveemos. Conviene facilitarle a Dafny los lemas de forma tal que el verificador pueda hacer \textit{pattern-matching} entre las variables que están en el scope donde el lema es necesario y sus parámetros de entrada.

    \chapter{Aplicando la metodología a nuevos problemas}

    \section{Suma del segmento de suma máxima}

    En este problema, el objetivo es averiguar la sumatoria de los elementos del segmento contiguo de suma máxima en un arreglo de números enteros.
    Como el arreglo puede contener números negativos, la solución no siempre es el segmento equivalente al total del arreglo.

    El problema del segmento de suma máxima fue resuelto por Joseph Kadane con un algoritmo de $\mathcal{O}(N)$ que analizaremos a continuación.

    \subsection*{El algoritmo de Joseph Kadane}
    Ilustraremos el arreglo considerando $N=4$ y utilizando cartas de póker puestas boca abajo, que vamos descubriendo una a una de izquierda a derecha. Utilizaremos cartas de corazón para representar números positivos, y cartas de trébol para números negativos.

    La idea es que en cada paso, tendremos una solución subóptima que será la suma de segmento máxima para los segmentos conocidos hasta el momento (con las cartas ya descubiertas), de modo que al descubrir todas las cartas, tendremos la suma del segmento de suma máxima para todo el arreglo.

    Veamos un caso concreto en el que hemos descubierto ya las primeras dos cartas y estamos a punto de descubrir la tercera (ver figura \ref{desc_ter_carta}).

    \ifUsePstPoker
        \begin{figure}[h]
            \centering
            \psset{framebg=beige}\crdsevh
            \psset{framebg=beige}\crdtwoh
            \psset{backcolor=red}\crdback
            \psset{backcolor=red}\crdback
            \rput(-4.8,-1){\textbf{\^}} % Positioning the ^ character

            \caption{Descubriendo la tercera carta} \label{desc_ter_carta}
        \end{figure}
    \fi

    El segmento de suma máxima conocido hasta el momento es el compuesto por las dos primeras cartas. Al descubrir la tercera pueden pasar dos cosas:
    \begin{itemize}
        \item O bien, la carta es de corazones y nos sirve para formar un nuevo segmento de suma máxima, pues incluirla resulta en un segmento con sumatoria mayor a la conocida hasta ahora.
        \item O bien, la carta es de trébol y entonces el segmento de suma máxima conocido hasta el momento es el que ya conocíamos en la iteración anterior.
    \end{itemize}

    Supongamos que descubrimos la tercera carta, resulta ser un dos de trébol y pasamos a descubrir la cuarta carta (ver figura \ref{desc_cuarta_carta_a}).

    \ifUsePstPoker
        \begin{figure}[h]
            \centering
            \psset{framebg=beige}\crdsevh
            \psset{framebg=beige}\crdtwoh
            \psset{framebg=beige}\crdtwoc
            \psset{backcolor=red}\crdback
            \rput(-1.4,-1){\textbf{\^}} % Positioning the ^ character

            \caption{Descubriendo la cuarta carta, con suma positiva} \label{desc_cuarta_carta_a}
        \end{figure}
    \fi

    \noindent En este caso el segmento de suma máxima conocido hasta el momento es el compuesto únicamente por las dos primeras cartas. Al descubrir la cuarta carta tenemos las siguientes posibilidades:
    \begin{itemize}
        \item Es de trébol y el segmento de suma máxima seguirá siendo el ya conocido.
        \item Es de corazón y su valor es mayor a dos, con lo cual tenemos un nuevo segmento de suma máxima, compuesto por todas las cartas descubiertas hasta el momento.
        \item Es de corazón pero su valor no es mayor a dos, con lo cual el segmento de suma máxima seguirá siendo el ya conocido.
    \end{itemize} 

    Supongamos que en cambio la tercera carta resultaba ser un diez de trébol y pasábamos a descubrir la cuarta carta (ver figura \ref{desc_cuarta_carta_b}).

    \ifUsePstPoker
        \begin{figure}[h]
            \centering
            \psset{framebg=beige}\crdsevh
            \psset{framebg=beige}\crdtwoh
            \psset{framebg=beige}\crdtenc
            \psset{backcolor=red}\crdback
            \rput(-1.4,-1){\textbf{\^}} % Positioning the ^ character

            \caption{Descubriendo la cuarta carta, con suma negativa} \label{desc_cuarta_carta_b}
        \end{figure}
    \fi

    \noindent En este caso el segmento de suma máxima conocido hasta el momento es el compuesto únicamente por las dos primeras cartas. Al descubrir la cuarta carta tenemos las siguientes posibilidades:
    \begin{itemize}
        \item Es de trébol y el segmento de suma máxima seguirá siendo el ya conocido.
        \item Es de corazón y su valor es mayor a nueve, con lo cual tenemos un nuevo segmento de suma máxima, compuesto únicamente por esta última carta.
        \item Es de corazón pero su valor no es mayor a nueve, con lo cual el segmento de suma máxima seguirá siendo el ya conocido.
    \end{itemize}

    Notemos que siempre que al descubrir una nueva carta logremos obtener un nuevo segmento de suma máxima  este será:
    \begin{itemize}
        \item O bien el segmento resultante de agregar la nueva carta al segmento que, de entre los segmentos que incluyen a la carta anterior, tenga suma máxima, en el caso de que alguno de ellos tenga una suma positiva como sucedería en la figura \ref{desc_cuarta_carta_a} si obtenemos una cuarta carta de corazón mayor a dos.
        \item O bien será el segmento compuesto por esta única carta, como podría suceder en la figura \ref{desc_cuarta_carta_b} si obtenemos una cuarta carta de corazón mayor a nueve.
    \end{itemize}

    En todo caso, si en cada iteración conocemos por un lado, la suma del segmento de suma máxima conocido hasta el momento y por otro, de entre los segmentos que incluyen a la carta anterior, la suma de aquel con suma máxima; y en particular conocemos esto en la última iteración, entonces tendremos toda la información necesaria para determinar el segmento de suma máxima del arreglo, habiéndolo recorrido una sola vez.

    Formalicemos ahora el problema y la estrategia resolutiva para obtener una implementación verificada en Dafny.

    \subsection*{Implementación en Dafny}
    Sea $A$ un arreglo de longitud $N$, nos referiremos con $A[p, q)$ al segmento cuya primera posición es $p$ y su última posición es $q-1$; de aquí se sigue que deberemos asumir $0 \leqslant p \leqslant q \leqslant |A|$.

    El segmento vacío denotado por dos índices idénticos (ejemplo: $A[i,i)$) tiene suma 0.

    El primer paso entonces es especificar el programa. En Dafny podemos utilizar funciones recursivas dentro de la especificación de los métodos. Definiremos la función \textit{suma} de manera recursiva para denotar la suma del segmento $A[p,q)$. 
    Y un predicado para la suma máxima que utilizamos como poscondición del método principal.

    \begin{greenbox}
    \begin{dafny}[gobble=8]
        function suma(p:nat, q:nat, A: seq<int>): int
            requires 0 <= p <= q <= |A|
        {
            if q <= p then 0 else suma(p, q-1, A) + A[q-1]
        }

        predicate es_suma_maxima(A: seq<int>, r: int) {
            (exists p, q :: 0 <= p <= q <= |A| && suma(p, q, A) == r) &&
            (forall i, j :: 0 <= i <= j <= |A| ==> suma(i, j, A) <= r)
        }

        method segmento_de_suma_maxima(A: seq<int>) returns (r: int)
            ensures suma_maxima(A, r)
    \end{dafny}
    \end{greenbox}

    A continuación escribimos algunos casos simples de prueba para ganar confianza en la especificación de la función \textit{suma} y el predicado \textit{es\_suma\_maxima}

    \begin{greenbox}
    \begin{dafny}[gobble=8]
        method test_suma()
        {
            assert suma(0, 3, [1,2,-3,4,5]) == 0;
            assert suma(0, 4, [1,2,-3,4,5]) == 4;
            assert suma(3, 5, [1,2,-3,4,5]) == 9;
            assert suma(4, 4, [1,2,-3,4,5]) == 0;
        }

        method test_es_suma_maxima()
        {
            assert suma(0, 4, [1,2,-2,4]) == 5;
            assert es_suma_maxima([1,2,-2,4], 5);
            assert es_suma_maxima([1,2,-3], 3);
            assert suma(0, 3, [1,2,3]) == 6;
            assert es_suma_maxima([1,2,3], 6);
        }
    \end{dafny}
    \end{greenbox}

    Ahora formalicemos un invariante, una guarda y una función de cota para el ciclo del programa. Utilizaremos $k$ para referenciar la iteración $k$-ésima, $A$ para el arreglo de entrada, $r$ para denotar la suma del segmento de suma máxima conocido hasta el momento.

    \begin{greenbox}
    \begin{dafny}[gobble=8]
        predicate invariante(k: int, A: seq<int>, r: int){
            0 <= k <= |A| &&
            (exists p, q :: 0 <= p <= q <= k && suma(p, q, A) == r) &&
            (forall i, j :: 0 <= i <= j <= k ==> suma(i, j, A) <= r) &&
        }
    \end{dafny}
    \end{greenbox}

    Notar que aún no hemos incluído en el invariante la suma de segmento máxima para segmentos que incluyan al último elemento conocido. Lo cual resultará necesario más adelante cuando intentemos verificar una implementación del cuerpo del ciclo que sostenga el invariante.

    Para la guarda postulamos $k < |A|$. Y para la función de cota $|A| - k$. Probemos ahora inicializar las variables con métodos de inicialización y cuerpo que aún no implementaremos pero especificaremos en relación al invariante y la guarda.

    \begin{greenbox}
    \begin{dafny}[gobble=8]
        method inicializacion(A: seq<int>) returns (k: int, r: int)
            ensures invariante(k, A, r)

        method cuerpo(k: int, A: seq<int>, r: int) returns (k': int, r': int)
            requires k < |A|
            requires invariante(k, A, r)
            ensures invariante(k', A, r')
            ensures k' > k

        method segmento_de_suma_maxima(A: seq<int>) returns (r: int)
            ensures es_suma_maxima(A, r)
        {
            var k: int;
            k, r := inicializacion(A);
            while (k < |A|)
                decreases |A| - k
                invariant invariante(k, A, r)
            {
                k, r := cuerpo(k, A, r);
            }
        }
    \end{dafny}
    \end{greenbox}

    La verificación del método \textit{segmento de suma maxima} resulta satisfactoria tras haber asumido que la inicialización y el cuerpo cumplen con su especificación. Podemos pasar ahora a implementarlos.

    \begin{greenbox}
    \begin{dafny}[gobble=8]
        method inicializacion(A: seq<int>) returns (k: int, r: int)
            ensures invariante(k, A, r)
        {
            k := 0;
            r := 0;
            assert suma(0, 0, A) == 0;
        }
    \end{dafny}
    \end{greenbox}

    Para la inicialización, la única ayuda que tuvimos que darle a Dafny es inducirlo a probar que la suma del segmento vacío $A[0,0)$ es igual a 0 (utilizando el assert de la línea 6). Con lo cual puede probar por sí mismo el resto del invariante.

    En cuanto nos disponemos a implementar el cuerpo notamos que no contamos con una variable que albergue la suma del segmento de suma máxima incluyendo el elemento anterior, que era necesario para computar el nuevo segmento de suma máxima.
    Utilizaremos $u$ para albergar la suma del segmento de suma máxima que incluye a $A[k-1]$, o del segmento vacío (con suma $0$) en caso de que no hayamos logrado acumular una suma positiva que incluya a $A[k-1]$. Por tanto, actualizamos el invariante con una cláusula sobre $u$ y con él la inicialización y la declaración de variables en el método principal.

    \begin{greenbox}
    \begin{dafny}[gobble=8]
        predicate invariante(k: int, A: seq<int>, r: int, u: int){
            0 <= k <= |A| &&
            (exists p_u :: 0 <= p_u <= k && suma(p_u, k, A) == u) &&
            (forall i :: 0 <= i <= k ==> suma(i, k, A) <= u) &&
            (exists p, q :: 0 <= p <= q <= k && suma(p, q, A) == r) &&
            (forall i, j :: 0 <= i <= j <= k ==> suma(i, j, A) <= r)
        }
    \end{dafny}
    \end{greenbox}

    Hemos agregado, de forma intencional, las nuevas cláusulas sobre $u$ antes de las cláusulas sobre $r$. Esto es por que sabemos que el verificador necesitará de las nuevas cláusulas para poder probar las cláusulas sobre $r$. Por tanto es buena idea presentárselas en ese orden. Si el verificador no puede probar las cláusulas para $u$ la verificación se detendrá allí y el mensaje de error será más acotado.

    Con el nuevo invariante postulamos la implementación del cuerpo del ciclo.

    \begin{whitebox}[before skip=2ex]
    \begin{dafny}[gobble=8]
        method cuerpo(k: int, A: seq<int>, r: int, u:int)
            returns (k': int, r': int, u': int)
            requires k < |A|
            requires invariante(k, A, r, u)
    \end{dafny}
    \end{whitebox}
    \begin{redbox}
    \begin{dafny}[gobble=8,firstnumber=5]
            ensures invariante(k', A, r', u')
    \end{dafny}
    \end{redbox}
    \begin{whitebox}[after skip=2ex]
    \begin{dafny}[gobble=8,firstnumber=6]
            ensures k' > k
        {
            if u + A[k] < 0 {
                u' := 0;
            } else {
                u' := u + A[k];
            }
            if u' > r {
                r' := u';
            } else {
                r' := r;
            }
            k' := k + 1;
        }
    \end{dafny}
    \end{whitebox}

    \begin{redbox}[after skip=2ex]
        This postcondition could not be proved on a return path.\\
        Could not prove: \inlinedafny{exists p_u :: 0 <= p_u <= k && suma(p_u, k, A) == u}
    \end{redbox}

    El primer \textit{if-else} define el valor de $u'$, que será el resultado de sumar el nuevo elemento a $u$ en caso de que esta suma resulte no negativa. De lo contrario debemos asignar a $u'$ la suma del segmento vacío.
    El segundo \textit{if-else} define el valor de $r'$, que será $u'$ en caso de que este supere al segmento de suma máxima conocido hasta ahora, o tendrá el valor de este último en caso contrario.

    El verificador en este caso falla en realizar la prueba al no poder probar que la implementación mantiene el invariante.
 
    Ayudemos al verificador en el razonamiento, como vimos al estudiar la regla del \textit{if}, podemos hacer el análisis caso por caso. En el caso $u + A[k] < 0$ sabemos que $suma(k+1, k+1, A) = u' = 0$. Mientras que en el caso $u + A[k] \geqslant 0$, tenemos que el mismo \inlinedafny{p_u} para el cual la cláusula se cumplía con $k$ y $u$ valdrá para $k'=k+1$ y $u'=u + A[k]$.
    Podemos inducir a Dafny a verificar estas condiciones mediante cláusulas \textit{assert} en cada caso del \textit{if-else}.

    \begin{greenbox}
    \begin{dafny}[gobble=8]
        if u + A[k] < 0 {
            u' := 0;
            assert suma(k+1, k+1, A) == u';
        } else {
            u' := u + A[k];
            ghost var p_u :| 
                (0 <= p_u <= k && suma(p_u, k, A) == u)
                && (forall i :: 0 <= i <= k ==> suma(i, k, A) <= u);
            assert suma(p_u, k+1, A) == u';
        }
    \end{dafny}
    \end{greenbox}

    Con esta ayuda, Dafny logra verificar el método cuerpo y consecuentemente la implementación de todo el programa, la cual recopilamos a continuación.

    \begin{greenbox}
        \dafnyfile{Implementación verificada de segmento de suma máxima}{ejemplos/calculo_de_programas/segmento_de_suma_maxima.dfy}
    \end{greenbox}

    \section{Un ejercicio de examen}
    Aplicaremos ahora la metodología para programar en Dafny el programa propuesto en el ejercicio 2 de \href{https://github.com/ExamenesViejos-FAMAF-Computacion/ExamenesViejos_AlgoritmosYEstructurasDeDatos1_FAMAF/blob/d2ef152c594b2847f8c95a0c423d333403aaa88b/Pr%C3%A1ctico/Final%202024-12-03.jpg}{uno de los exámenes finales} de Algoritmos y Estructura de Datos I del año 2024. El cual transcribimos aquí:

    Considere el problema especificado de la siguiente manera:
    \begin{align*}
        &Const\ N: Int;\\
        &Var\ A:\ array\ [0, N)\ of\ Int;\\
        &Var\ r:\ Bool;\\
        &\{N \geqslant 0\}\\
        &    S\\
        &\{r = \langle\forall i : 0 \leqslant i \leqslant N : \langle\sum j : 0 \leqslant j < i \land A.j\ \text{mod}\ 2 = 1 : A.j\rangle \leqslant i! \rangle \}\\
    \end{align*}

    \begin{enumerate}
        \item Derivar un programa imperativo que resuelva este problema. El programa \textbf{debe recorrer una sola vez el arreglo} (sin ciclos anidados).
        \item No puede usarse la función factorial en el programa, será necesario mantener un invariante adecuado.
        \item Optimizar el ciclo fortaleciendo la guarda para que el programa termine si se hace falso el resultado. Demostrar que el invariante y la nueva guarda negada implican la postcondición.
    \end{enumerate}

    \subsection*{Argumento}
    En este ejercicio debemos obtener un programa que dado un arreglo de $N$ números enteros, establezca en una variable booleana $r$ si para todas las posiciones $i$ del arreglo la sumatoria de los números impares precedentes a esa posición (sin incluir el número en dicha posición) es menor o igual a $i$ factorial. Además debemos hacerlo recorriendo una sola vez el arreglo y sin utilizar la función factorial.

    La clave para resolver este problema es notar dos cosas. Por un lado, si al momento de recorrer la posición $i$ tenemos en una variable $sum\_imp$ la sumatoria de los números impares precedentes a la posición $i$ entonces la sumatoria de los números impares precedentes a $i+1$ será $sum\_imp + A.i$ en caso de que $A.i$ sea impar o simplemente $sum\_imp$ en caso de que sea par.
    A su vez si tenemos en una variable $fact\_i$ el factorial de $i$, entonces el factorial de $i+1$ será  $fact\_i * (i+1)$.

    Con estas dos verdades, proponemos como estrategia iterar el arreglo de izquierda a derecha y mantener en cada iteración $k$ el invariante:

    \begin{align*}
        Inv: \{r = \langle\forall i : 0 \leqslant i \leqslant k : \langle\sum j : 0 \leqslant j < k \land A.j\ \text{mod}\ 2 = 1 : A.j\rangle \leqslant i! \rangle \}
    \end{align*}

    Avanzaremos, hasta terminar de recorrer el arreglo o bien hasta que $r$ se torne falso. Es decir, la guarda será $r \land k < |A|$, y la función de cota $|A| - 1$.
    Claro que, para lograr mantener ese invariante, necesitaremos fortalecerlo con invariantes para $sum\_imp$ y $fact\_i$. Pero esto lo haremos cuando intentemos implementar el cuerpo del ciclo.

    Pasemos ahora a implementar el programa en Dafny siguiendo la metodología.

    \subsection*{Implementación en Dafny}

    \subsubsection*{Escribir la especificación}
    Para especificar este programa necesitaremos definir dos funciones de manera recursiva en Dafny: \inlinedafny{sum_imp_f(A, i)} para expresar la sumatoria de los números impares en $A$ precedentes a la posición $i$. Y \inlinedafny{fact_i(i)} para expresar el factorial de $i$. Estas funciones serán utilizadas por Dafny para realizar la verificación pero no forman parte de la ejecución del programa.

    Recordemos que es importante \textit{testear} las funciones para ganar confianza en que expresan lo que deseamos que expresen.

    \begin{greenbox}
    \begin{dafny}[gobble=8]
        function sum_imp_f(A: seq<int>, i: int): int
            requires 0 <= i <= |A|
        {
            if i <= 0
                then 0
                else (
                    if A[i-1] % 2 == 1
                        then A[i-1] + sum_imp_f(A, i - 1)
                        else sum_imp_f(A, i -1)
                )
        }

        method test_sum_imp_f()
        {
            assert sum_imp_f([1,2,3,4,5], 0) == 0;
            assert sum_imp_f([1,2,3,4,5], 1) == 1;
            assert sum_imp_f([1,2,3,4,5], 2) == 1;
            assert sum_imp_f([1,2,3,4,5], 3) == 4;
            assert sum_imp_f([1,2,3,4,5], 4) == 4;
            assert sum_imp_f([1,2,3,4,5], 5) == 9;
        }

        function factorial_f(i: int): int
            requires i >= 0
        {
            if i == 0 then 1 else i * factorial_f(i - 1)
        }

        method test_factorial_f() {
            assert factorial_f(0) == 1;
            assert factorial_f(1) == 1;
            assert factorial_f(2) == 2;
            assert factorial_f(3) == 6;
            assert factorial_f(4) == 24;
        }
    \end{dafny}
    \end{greenbox}

    Utilizando estas funciones podemos expresar la poscondición $Q$ como un predicado sobre $r$ y $A$, y \textit{testearlo} para algunos casos pequeños.

    \begin{greenbox}
    \begin{dafny}[gobble=8]
        predicate Q(r: bool, A: seq<int>)
        {
            r == (forall i: int :: 0 <= i <= |A| 
                    ==> sum_imp_f(A, i) <= factorial_f(i))
        }

        method test_Q(){
            assert Q(true, [1,2,3,4,5]);
            assert factorial_f(1) < 7;
            assert Q(false, [7,2,3,4,5]);
        }
    \end{dafny}
    \end{greenbox}

    Para el segundo caso de \textit{test} fue necesario inducir a Dafny a probar primero que el factorial de $1$ es menor a $7$.

    Y finalmente especificar el método principal utilizando este predicado:

    \begin{greenbox}
    \begin{dafny}[gobble=8]
        method ejercicio(A: seq<int>)
            returns (r: bool)
            ensures Q(r, A)
    \end{dafny}
    \end{greenbox}

    \subsubsection*{Definir el invariante, guarda y función de cota}
    Esto ya lo hicimos en la sección ``Argumento'', lo que haremos ahora será escribir el invariante que habíamos definido allí como un predicado de Dafny.
    Nos será útil generalizar el predicado de la poscondición $Q$ en un predicado $Q\_k$ (cambio de constante por variable).
    El cual podremos reutilizar tanto para redefinir $Q$ como para definir el invariante.

    \begin{greenbox}
    \begin{dafny}[gobble=8]
        predicate Q_k(r: bool, A: seq<int>, k: int)
        {
            0 <= k <= |A|
            && r == (forall i: int :: 0 <= i <= k
                        ==> sum_imp_f(A, i) <= factorial_f(i))
        }
        
        predicate Q(r: bool, A: seq<int>)
        {
            Q_k(r, A, |A|)
        }
        
        predicate invariante(r: bool, A: seq<int>, k: int)
        {
            Q_k(r, A, k)
        }
    \end{dafny}
    \end{greenbox}

    \subsubsection*{Esqueleto del método principal}
    Ahora pasaremos a escribir la declaración de variables, la especificación de la inicialización y el cuerpo del ciclo y la estructura del método principal todo en un solo paso.

    \begin{greenbox}
    \begin{dafny}[gobble=8]
        method inicializacion(A: seq<int>)
            returns (i: int, r: bool)
            ensures invariante(r, A, i)

        method cuerpo(r:bool, A: seq<int>, i: int)
            returns (r': bool, i': int)
            requires r && i < |A|
            requires invariante(r, A, i)
            ensures invariante(r', A, i')
            ensures i' > i

        method ejercicio(A: seq<int>)
            returns (r: bool)
            ensures Q(r, A)
        {
            var i: int;
            i, r := inicializacion(A);
            while r && i < |A|
                decreases |A| - i
                invariant invariante(r, A, i)
            {
                r, i := cuerpo(r, A, i);
            }
            return r;
        }
    \end{dafny}
    \end{greenbox}

    Dafny logra verificar el método principal, con lo cual validamos que hemos elegido bien la guarda, el invariante y la función de cota. Y podemos pasar a la implementación de la inicialización y el cuerpo del ciclo.

    \subsubsection*{Implementación de inicialización y cuerpo del ciclo}

    Para la inicialización de $i$ la variable de iteración elegimos $0$, la primera posición del arreglo. Para $r$ elegimos $true$ para equiparar la valuación del cuantificador universal con rango vacío.

    \begin{greenbox}
    \begin{dafny}[gobble=8]
        method inicializacion(A: seq<int>)
        returns (i: int, r: bool)
        ensures invariante(r, A, i)
    {
        i := 0;
        r := true;
    }
    \end{dafny}
    \end{greenbox}

    Dafny verifica la corrección de la inicialización. Al intentar implementar el cuerpo del ciclo, recordamos necesarias las variables $sum\_imp$ y $fact\_i$ que planificamos durante el argumento. Con lo cual pasamos primero a incorporarlas en el invariante.

    \begin{greenbox}
    \begin{dafny}[gobble=8]
        predicate invariante(r: bool, A: seq<int>, k: int, sum_imp: int, i_fact: int)
        {
            0 <= k <= |A|
            && sum_imp == sum_imp_f(A, k)
            && i_fact == factorial_f(k)
            && Q_k(r, A, k)
        }
    \end{dafny}
    \end{greenbox}

    Aquí establecemos la relación entre estas variables y los valores que queríamos mantener en ellas. $sum\_imp$ será la suma de los números impares en $A$ precedentes a la posición $k$; mientras que $i\_fact$ será el factorial de $k$.
    Además necesitamos anteponer la condición sobre el rango en el que se mueve $k$ para cumplir con las precondiciones de las funciones $sum\_imp\_f$ y $factorial\_f$.

    Luego pasamos a actualizar las variables en el método principal, la inicialización y la firma del cuerpo.

    \begin{greenbox}
    \begin{dafny}[gobble=8]
        method inicializacion(A: seq<int>)
            returns (i: int, r: bool, sum_imp: int, i_fact: int)
            ensures invariante(r, A, i, sum_imp, i_fact)
        {
            i := 0;
            r := true;
            sum_imp := 0;
            i_fact := 1;
        }
        
        method cuerpo(r:bool, A: seq<int>, i: int, sum_imp: int, i_fact: int)
            returns (r': bool, i': int, sum_imp':int , i_fact':int)
            requires r && i < |A|
            requires invariante(r, A, i, sum_imp, i_fact)
            ensures invariante(r', A, i', sum_imp', i_fact')
            ensures i' > i


        method ejercicio(A: seq<int>)
            returns (r: bool)
            ensures Q(r, A)
        {
            var i: int, sum_imp: int, i_fact: int;
            i, r, sum_imp, i_fact := inicializacion(A);
            while r && i < |A|
                decreases |A| - i
                invariant invariante(r, A, i, sum_imp, i_fact)
            {
                r, i, sum_imp, i_fact := cuerpo(r, A, i, sum_imp, i_fact);
            }
        }
    \end{dafny}
    \end{greenbox}

    Solo queda implementar el cuerpo siguiendo el razonamiento que hicimos durante el argumento.

    \begin{greenbox}
    \begin{dafny}[gobble=8]
        method cuerpo(r:bool, A: seq<int>, i: int, sum_imp: int, i_fact: int)
            returns (r': bool, i': int, sum_imp':int , i_fact':int)
            requires r && i < |A|
            requires invariante(r, A, i, sum_imp, i_fact)
            ensures invariante(r', A, i', sum_imp', i_fact')
            ensures i' > i
        {
            if A[i] % 2 == 1 {
                sum_imp' := sum_imp + A[i];
            } else {
                sum_imp' := sum_imp;
            }
            i_fact' := (i + 1) * i_fact;
            r' := sum_imp' <= i_fact';
            i' := i + 1;
        }
    \end{dafny}
    \end{greenbox}

    Aquí la actualización de $r$ es \inlinedafny{r' := sum_imp' <= i_fact';} ya que la por la conjunción de la guarda y el invariante sabemos que $Q(r, A, k)$ vale, y por tanto $Q(r', A, k+1)$ dependerá solamente de esta última comparación.

    Dafny logra verificar el cuerpo del ciclo sin intervención alguna de nuestra parte y con ello todo el programa, el cual mostramos a continuación de manera unificada:

    \begin{greenbox}
    \dafnyfile{Ejercicio de examen final 2024 verificado en Dafny}{ejemplos/final_2024/programa_completo.dfy}
    \end{greenbox}
    \chapter{Breve mirada al interior de Dafny} \label{ch:pipeline-dafny}

    En este capítulo intentaremos entender a grandes razgos cómo funciona Dafny internamente y cómo llegó a ser lo que es, desde una perspectiva histórica.

    \section{Dafny: ¿De qué planeta viniste?}
    Durante la década de los setenta y los ochenta, a la base axiomática para la programación imperativa presentada por Hoare\cite{10.1145/363235.363259} le siguieron investigaciones sobre la generación de condiciones de verificación (VCs, por sus siglas en inglés) para instrumentar la prueba de corrección de un programa. Dado un programa y su especificación, sus condiciones de verificación son fórmulas en lógica de primer órden que, si todas resultan válidas, entonces sabemos que existe una demostración en la lógica de Hoare de que el programa es correcto respecto de su especificación, es decir que $\vdash[P]S[Q]$\footnote{Este tópico está muy bien explicado en las notas de \href{https://www.cl.cam.ac.uk/archive/mjcg/HL/Notes/Notes.pdf}{Mike Gordon sobre la Lógica de Hoare}}. Uno de los esfuerzos icónicos en ese sentido es el cálculo de Weakest Preconditions (WP) presentado por Dijkstra\cite{EWD:EWD418}, el cual introduce una mecánica para obtener VCs de manera composicional a partir de la definición de la precondición más débil de cada sentencia del programa respecto a una poscondición deseada.
    La generación de VCs facilita la prueba de la corrección de programas por que reduce esta tarea a la de demostrar la validez de fórmulas de lógica de primer orden. Esto permite aprovechar la experiencia en la manipulación del sistema formal lógico para realizar pruebas manuales, pero también posibilita la delegación de estas condiciones de verificación a probadores automáticos de teoremas, capaces de determinar si son válidas o existen contraejemplos.

    Por otra parte, durante la década de los noventa, los principales protagonistas en la creación de Dafny, entonces investigadores del Compaq Systems Research Center, llevaron a cabo los primeros experimentos en ``Extended Static Checkers (ESC)'' desarrollados para el lenguaje Módula-3 (ESC/Modula-3) y Java (ESC/Java)\cite{Leino2001}.
    En un principio los ESC buscaron ampliar la capacidad del chequeo estático a un conjunto de errores más amplio que los errores de tipado y que, hasta entonces, solo podían ser detectados en tiempo de ejecución, por ejemplo: intentos de acceder a atributos en una variable nula, índices fuera de rango, división por cero, errores de casteo de tipos, condiciones de carrera, deadlocks, etc. Luego se volcaron al objetivo de verificar que un programa anotado con precondiciones y poscondiciones cumpla con su especificación.

    Los ESC utilizaron desde un principio tecnología de verificación de programas, es decir, agregaban anotaciones a los programas, y generaban condiciones de verificación, que podían probarse si y solo si el programa estaba libre del tipo de errores que se buscaba detectar. Estas condiciones de verificación eran delegadas luego a un probador automático de teoremas.

    Al tener como objetivo mejorar la detección temprana de errores en proyectos de desarrollo reales, los ESC se enfrentaron con varios desafíos prácticos. Algunos de ellos fueron: decidir qué tipos de errores buscarían detectar (teniendo en cuenta el \textit{trade off} entre cobertura y esfuerzo), formalizar semánticamente lenguajes de programación modernos, utilizar eficientemente probadores automáticos de teoremas, producir reportes de errores útiles a partir del resultado de los probadores de teoremas, decidir qué anotaciones harían falta en los programas y convencer a los programadores de llevarlas a cabo.

    De la experiencia de asumir estos desafíos se desprendieron algunas ideas claves que se mantienen hasta el día de hoy y forman parte de la arquitectura de solución de Dafny. Algunas de ellas son:
    \begin{itemize}
        \item Resolver la formalización de la semántica de lenguajes modernos en dos etapas. Primero, desarrollando un lenguaje intermedio de verificación (IVL, por sus siglas en inglés), el cual está específicamente diseñado y acotado para obtener condiciones de verificación. Y luego traducir los distintos lenguajes a este lenguaje intermedio, acompañados de un \textit{background  predicate} que formaliza las propiedades de los operadores, tipos, y demás construcciones del lenguaje que no tengan una traducción directa en el IVL. Este \textit{background  predicate} o preludio, es luego incorporado en la condición de verificación en conjunción con la precondición.
        \item Priorizar la generación de condiciones de verificación que maximicen la eficiencia del probador de teoremas. Existen múltiples formas matemáticamente equivalentes de generar el conjunto de condiciones de verificación. Pero a la hora de integrar la verificación automática dentro del proceso de desarrollo es necesario que las VCs generadas optimicen el tiempo que tardará el probador de teoremas en verificarlas. Esto depende principalmente de su tamaño, para lo cual es necesario evitar la redundancia.
    \end{itemize}

    La experiencia en el desarrollo de los Extended Static Checkers sumado al desarrollo de potentes probadores de teoremas de tipo satisfiability-modulo-theories (SMT Solvers), más recientemente, permitió que contemos hoy con verificadores automáticos de software como Dafny.

    \section{El pipeline de verificación en Dafny}
    El pipeline de verificación de programas Dafny se compone principalmente, por tres etapas que explicaremos en esta sección:
    \begin{itemize}
        \item La traducción del programa Dafny a un programa en el IVL Boogie\footnote{
            Boogie fue escrito originalmente como herramienta intermedia del proyecto Spec\#, una extensión de C\# con soporte para especificaciones. Luego tomó vida propia y hoy es utilizado para construir verificadores para otros lenguajes, como el verificador de Dafny para el lenguaje del mismo nombre, o VCC y HAVOC, dos verificadores de C.
        }.
        \item La generación de condiciones de verificación a partir del programa Boogie.
        \item La descarga de esas condiciones de verificación en un SMT Solver, típicamente el Z3 Solver.
    \end{itemize}

    \subsection{La traducción de Dafny a Boogie}
    La traducción de Dafny a Boogie no es trivial, la pieza de código que se encarga de ello tiene aproximadamente un millón de líneas escritas en C\#, y busca codificar cada construcción sintáctica del lenguaje de Dafny en una construcción equivalente de Boogie.\footnote{El código fuente para la traducción se encuentra en el \href{https://github.com/dafny-lang/dafny/tree/v4.7.0/Source/DafnyCore/Verifier}{directorio ``Verifier''} del repositorio de Dafny, en donde la clase ``BoogieGenerator'' se construye modularmente a través de múltiples archivos.} Para aquellos tipos y operadores que no pueden ser codificados en Boogie, Dafny provee un preludio\footnote{El código fuente del preludio se encuentra apartado en el \href{https://github.com/dafny-lang/dafny/blob/v4.7.0/Source/DafnyCore/DafnyPrelude.bpl}{directorio ``Prelude''} del repositorio de Dafny} que los formaliza.

    Utilizando la línea de comandos de Dafny podemos obtener la traducción a código Boogie de la siguiente manera
    \begin{verbatim}
        $ dafny verify my_program.dfy --bprint:my_program.bpl
    \end{verbatim}
    Al hacerlo, nos encontraremos con un programa Boogie mucho más extenso que nuestro programa original en Dafny. Esto es porque la traducción empieza con el preludio de Dafny seguido de la traducción en sí de nuestro programa.

    \subsection{La generación de condiciones de verificación}
    La generación de condiciones de verificación consiste en una serie de transformaciones del programa Boogie inicial hacia una versión tal que permita la construcción de una fórmula lógica a partir de las sentencias del programa de forma directa.

    Este pipeline fue descripto por Barnett y Leino\cite{10.1145/1108792.1108813} y se recomienda su lectura para entenderlo en detalle. Aquí haremos un repaso de los conceptos principales que allí se describen.

    \subsubsection*{El lenguaje no estructurado de partida}
    El programa resultante de la traducción de Dafny a Boogie se somete luego a una transformación hacia un lenguaje no estructurado siguiendo la siguiente gramática:
    \begin{align*}
        Program \;\;&::=\;\; Block^{+} \\
          Block \;\;&::=\;\; BlockId :\; Stmt;\;\textbf{goto } BlockId^{*} \\
           Stmt \;\;&::=\;\; VarId := Expr\;|\;\textbf{havoc } VarId \\
                &\;\;\;\;\;\;\;\;|\ \textbf{assert } Expr\;|\;\textbf{assume } Expr \\
                &\;\;\;\;\;\;\;\;|\ Stmt ; Stmt \;|\; \textbf{skip}
    \end{align*}
    En este pequeño lenguaje un programa está compuesto por uno o más bloques. Cada bloque está identificado por un $BlockId$, tiene un cuerpo compuesto por una o más sentencias y un conjunto de cero o más bloques \textit{sucesores} (los listados luego del \textbf{goto}). Al bloque que inicia la ejecución del programa lo etiquetamos con el identificador $Start$. Luego de ejecutar el cuerpo de un bloque se continúa arbitrariamente con alguno de los bloques sucesores.

    El operador ``$:=$'' denota la asignación, $\textbf{havoc}$ la asignación de un valor arbitrario a una variable. El operador ``$;$'' se utiliza para componer. Las sentencias $\textbf{assert}$ y $\textbf{assume}$ tienen un rol clave en la generación de condiciones de verificación y $\textbf{skip}$ es simplemente un atajo para $\textbf{assert True}$.

    La transformación del programa estructurado inicial se logra reemplazando ciclos y condicionales por estructuras equivalentes en esta gramática. Los condicionales de la forma:
    \begin{align*}
        \textbf{if}\ (E)\ \{S\}\ \textbf{else}\ \{T\}
    \end{align*}
    se reemplazan con:
    \begin{align*}
        Start:&\;\;\;\textbf{skip};\ \textbf{goto}\ Then,\ Else \\
        Then:&\;\;\;\textbf{assume}\ E;\ S;\ \textbf{goto}\ End \\
        Else:&\;\;\;\textbf{assume}\ \lnot E;\ T;\ \textbf{goto}\ End \\
        End:&\;\;\;...
    \end{align*}

    \noindent y los ciclos se reemplazan con \textbf{goto's} al bloque cabecera del ciclo, de la siguiente manera:

    \begin{align*}
        LoopHead:\;\;\;&\textbf{assert}\ \text{loop invariant};\\
                       &\textbf{goto}\ Body,\ After \\
        Body:\;\;\;&\textbf{assume}\ \text{loop guard};\\
                   &S;\\
                   &\textbf{goto}\ LoopHead \\
        After:\;\;\;&\textbf{assume}\ \lnot \text{loop guard};\\
                    & ...
    \end{align*}

    Las precondiciones pueden codificarse incorporando sentencias $\textbf{assume}$ al principio del bloque $Start$ y las postcondiciones como sentencias $\textbf{assert}$ al final de los bloques sin sucesores.
    Cada programa da lugar a un conjunto de trazas de ejecución posibles empezando desde el bloque $Start$. Un programa es correcto si ninguna de ellas contiene una sentencia $\textbf{assert}$ que evalúe a $False$.

    \subsubsection*{La eliminación de ciclos}
    A partir de una versión no estructurada del programa siguiendo esta gramática se prosigue a eliminar los ciclos, quitando los ``back edges'' (introducidos por las sentencias de tipo $\textbf{goto}\ LoopHead$). Para esto, primero se mueve la aserción del invariante al bloque predecesor de $LoopHead$, reemplazándolo por sentencias $\textbf{havoc}$ sobre las variables que pueden ser modificadas por el ciclo, seguido de una sentencias $\textbf{assume}$ para el invariante. De esta forma las trazas que pasen por el bloque $Body$ representan cualquier iteración arbitraria del ciclo y la sentencia $\textbf{goto}\ LoopHead$ puede eliminarse del cuerpo.
    Notar que esta transformación denota una fuerte dependencia en la definición de invariantes para los ciclos.

    \subsubsection*{La pasivización o eliminación de las asignaciones}
    Una vez eliminados los ciclos, se realiza la ``pasivización'' del programa que consiste en eliminar las asignaciones y en su lugar utilizar sentencias $\textbf{assume}$. Para ello primero se introducen variables auxiliares para cada ``encarnación'' o asignación de las variables del programa, asegurando que en cualquier traza de ejecución posible cada variable sea asignada una única vez, lo que permite a continuación transformar estas asignaciones únicas en sentencias $\textbf{assume}$.

    \subsubsection*{La construcción de la obligación de prueba}
    Concluídas estas dos transformaciones, se obtiene un programa compuesto por bloques de la forma:
    \begin{verbatim}
        A: S; goto ...
    \end{verbatim} 
    Donde $S$ solo puede ser $\textbf{assume}$, $\textbf{assert}$ o composición de estos dos.\footnote{Las sentencias de tipo $\textbf{havoc}$ pueden eliminarse a los fines de generar la condición de verificación, pues ló único que interesa sobre las variables son sus condicionantes expresados en las sentencias de tipo $\textbf{assume}$ y $\textbf{assert}$. Las sentencias de tipo $\textbf{skip}$, recordemos, son un atajo a $\textbf{assert}\ True$.}
    La Weakest Precondition de cada una de estos tipos de sentencias puede definirse como:
    \begin{align*}
        wp(\textbf{assert}\ P,\;Q)   &= P \land Q \\
        wp(\textbf{assume}\  P,\; Q) &= P \Rightarrow Q \\
        wp(S;T,\;Q)                  &= wp(S,\;wp(T,\;Q)) \\
    \end{align*}
    Para cada bloque $A$, se define la variable auxiliar $A_{ok}$. Intuitivamente $A_{ok}$ es $True$ si el programa está en un estado a partir del cual todas las ejecuciones que empiecen desde $A$ son correctas. Se postula la ``ecuación de bloque'' $A_{be}$ para definir $A_{ok}$ formalmente:
    \begin{equation}
        \label{eq:bloq-equation}
        A_{be}: A_{ok} \equiv wp(S, \bigwedge_{B \in Succ(A)} B_{ok})
    \end{equation}
    Donde $Succ(A)$ es el conjunto de bloques sucesores de $A$.

    Cada bloque aporta una ecuación en la forma de \ref{eq:bloq-equation}. Sea $R$ como la conjunción de las ecuaciones de bloque, la obligación de prueba del programa queda definida como:
    \begin{align}
        R \Rightarrow Start_{ok} \label{eq:VC}
    \end{align}

    Se puede ver que si el probador de teoremas resuelve \textit{sat} para la negación de la fórmula \ref{eq:VC}:
    \begin{align*}
        R \land \lnot Start_{ok}
    \end{align*}
    \noindent estamos en situación de que existe un estado a partir del cual, establecida la semántica del programa ($R$), hay una traza de ejecución incorrecta, es decir que el programa no cumple con su especificación.

    Este procedimiento para la generación de la condición de verificación que realiza Boogie busca crear fórmulas lógicas que sean acotadas en tamaño y eviten redundancia para optimizar el tiempo de respuesta del probador de teoremas.
  
    Utilizando la línea de comandos de Boogie podemos inspeccionar las distintas versiones del programa obtenidas tras realizar cada una de las etapas del pipeline con la opción \verb|/traceverify|
    \begin{verbatim}
        $ boogie /traceverify ejemplos/my_program.bpl
    \end{verbatim}

    \subsubsection*{La delegación de las condiciones de verificación}
    Las condiciones de verificación, que resultan del proceso descripto en la sección anterior, deben ser despachadas al probador de teoremas.
    Con la opción \verb|/proverLog| podemos ver el registro de la consulta realizada por Boogie al probador de teoremas, en formato de expresiones SMT-LIB y, como comentarios, las respuestas obtenidas:

    \begin{verbatim}
        $ boogie /proverLog:my_program.smt my_program.bpl
    \end{verbatim}

    Una de las tareas importantes que realizan Boogie y Dafny es abstraernos de la complejidad de este pipeline y mantener la trazabilidad entre los reportes del SMT Solver y el código fuente en Dafny utilizando etiquetas sobre los predicados de las VCs para poder indicarnos luego, en el editor de texto, el origen de los errores de verificación.

    \chapter{Conclusiones}
    Dafny es un lenguaje de programación accesible, con una sintaxis fácil de interpretar. La experiencia de escribir programas Dafny en un entorno de desarrollo que integra la verificación automática es enriquecedora. El verificador es rápido y el reporte de errores es claro en sus mensajes y preciso a la hora de indicar en qué segmento del código fuente hubo una condición de verificación que no pudo resolverse.

    Para que la interacción entre quien desarrolla y el verificador fluya y favorezca la comprensión, es necesario tener una metodología a la hora de escribir los programas. Presentamos una metodología que ordena el desarrollo y cuyo principio fundamental es particionar el problema en subproblemas más simples (esqueleto del método principal, cuerpo del ciclo, inicialización). Esta metodología permite compartimentar la interacción con el verificador, evitando escenarios donde los reportes de errores sean potencialmente numerosos y variados y nos resulte confuso identificar el origen de los mismos.

    Propusimos narrativas para explicar algoritmos imperativos simples, típicamente presentes en el dictado de la materia Algoritmos y Estructura de Datos I, mediante analogías entre dichos algoritmos y procedimientos interactivos del ser humano con objetos del mundo real (por ejemplo, el uso de ``varillas'' para entender el algoritmo de máximo común divisor de Euclides, o de ``cartas'' para el algoritmo de segmento de suma máxima de Kadane). Estas explicaciones resultaron efectivas para entender la estrategia resolutiva en cada caso y despejar los invariantes y funciones de cota que le dan sustento. Mostramos además que complementando estas narrativas con la comprensión de los axiomas y reglas de inferencia de la lógica de Hoare y una metodología para ordenar el desarrollo, se pueden lograr implementaciones verificadas en Dafny de los algoritmos expuestos.

    Estos elementos en su conjunto proponen un acercamiento a la programación formal que puede entenderse como alternativo o complementario a la propuesta académica de la materia. Creemos que sería de gran valor consolidar esta propuesta en un curso concreto que permita ponerla a prueba y medir resultados de aprendizaje en el estudiantado. Una hipótesis a validar sería que los estudiantes logren, hacia el final del curso, estar en mejores condiciones para explicar los algoritmos aprendidos, proponer algoritmos para problemas nuevos inspirados en los ya conocidos, y obtener implementaciones verificadas por computadora de los mismos.

    Por último, consideramos que delegar la mecanización de las pruebas en un verificador automático permite explorar en el mismo período de tiempo una mayor variedad de programas (por ejemplo, programas con arreglos, punteros y clases), y esto puede servir como puente para que los estudiantes sostengan el hábito de la programación formal en proyectos de software de mediano o gran tamaño. Dafny tiene soporte para múltiples tipos y estructuras de datos que no llegamos a visitar durante este trabajo. Consideramos que ampliar la propuesta de enseñanza presentada aquí para que los abarque es una línea de investigación futura interesante.

    \bibliography{References}
    \bibliographystyle{plain}
\end{document}
